%!TEX root = ../../thesis.tex
\section{Interval Arithmetic}
	When using floating point arithmetics several real numbers are mapped to the same floating point representation.
	A single floating point number does not contain any information about its accuracy.

	To put bounds on the error, one can instead work with sets of interval, instead of working on sets of single real numbers.
	This means a real number $r \in \RR$ is represented by an interval $x := [x_l,x_r]$, so that $r \in x$.
	The advantage to floating point arithmetics is that the length of the interval gives a bound on the rounding error, 
	thus making it possible to guarantee the accuracy of computed results.

	The fundamental property of interval arithmetic is the inclusion property:

	Every $f: \RR \to \RR$ can be extended to a function $\bar f$ on intervals, such that for all $x \in [a,b]$, $f(x) \in \bar f[a,b]$ 


	In general applying a function $f: \RR^k \to \RR$ on on intervals $x_1, \dots, x_k$ can be defined as
	\begin{equation}
		f(x_1, \dots, x_k) = [\min(\{f(r_1, \dots, r_n) \,|\, r_1 \in x_1, \dots, r_k\in x_k \}), \max(\{f(r_1, \dots, r_n) \,|\, r_1 \in x_1, \dots, r_k\in x_k \})]
	\end{equation}
	The basic arithmetic operations can be easily defined on intervals:
	\begin{theorem}
		Let $x = [x_l, x_r]$ and $y = [y_l, y_r]$ then it holds
		\begin{eqnarray}
			x + y  & = & [x_l + y_l, x_r + y_r] \\
			x - y  & = & [x_l - y_r, x_r + y_l] \\
			x \times y  & = & [\min(x_ly_l, x_ly_r, x_ry_l, x_ry_r), \max(x_ly_l, x_ly_r, x_ry_l, x_ry_r)] \\
			\frac{1}{x} & = & \left[\frac{1}{x_r}, \frac{1}{x_l} \right] \text{ if } x_l > 0 \text{ or } x_r < 0 \\
			\frac{x}{y} & = & x \times \frac{1}{y}   
		\end{eqnarray}
	\end{theorem}
	\begin{theorem}
		For every monotonic function $f: \RR \to \RR$ it holds
		$$ f(x) = [\min(f(x_l), f(x_r)), \max(f(x_l), f(x_r))] $$
		In particular
		\begin{equation}
			x^n  =   
				\begin{cases} 
					[x_l^n, x_r^n] &\mbox{if } n \mbox{ is even or } x_l \geq 0 \\
					[x_r^n, x_l^n] &\mbox{if n is odd and } x_r \leq 0 \\
					[0, \max(x_r^n, x_l^n)] & \mbox{otherwise.}  \\
				\end{cases} 
		\end{equation} 
	\end{theorem}
	In a similiar way, interval version for $\exp, \log, \sin, \cos$ and other elementary functions can be found, 
	that can then be used to find the interval versions of more complicated functions.

	The strength of interval arithmetic is, that it can easily be implemented even when the above 
	operations on the interval endpoints are not computed exactly, but finite approximations are used.
	
	For that to work, the only thing that is important, is that outward rounding is used, i.e. 
	after every operation the left point of the interval has to be rounded down and the right point rounded up.

	IEEE 754 requires, that the user is able to specify the rounding mode, thus interval arithmetic can be implemented 
	using standard floating point numbers.

	Note that the length of the interval can increase quickly  (example?)

	There are many implementations of interval arithmetic in different programming languages, e.g. 
	for \cc the boost interval arithmetic library can be used.

