  \subsection{Introduction}
 
  A chaotic system is a dynamical system, that is highly sensitive to initial conditions.
  A consequence of that is that even relatively small numerical errors in the computation grow exponentially fast.
  A prototypical example of a chaotic system is the logistic map:
  \begin{definition}\label{def:logmap}
  The \textbf{logistic map} is given by the recurrence relation 
  $ x_{n+1} = ax_n(1-x_n) \text{ with } a, x_0 \in \R$.
  \end{definition} 

 \subsubsection{The Shadowing Lemma}
 The term pseudo-orbit is used to describe numerically generated noisy orbits. 
 \begin{definition}\label{def:pseudoorbit}
 A sequence $(x_i)$ is called an \textbf{$\alpha$-pseudo-orbit} for a map $f$ if
 $ \| x_{i+1} - f(x_i) \| < \alpha $  
 \end{definition}

  \begin{definition}\label{def:shadowing}
	A real orbit $(y_i)$ \textbf{$\beta$-shadows} the pseudo-orbit $(x_i)$ if 
	$\| x_i - y_i \| < \beta$.  
 \end{definition}
  For systems that are uniformly hyperbolic Anosov and Bowen could show the following result:
  \begin{lemma}[Shadowing Lemma]
  For all $\beta > 0$ there exists an $\alpha > 0$ so that for every $\alpha$-pseudo-orbit $(x_i)$ there is a point $y_0$ so that the real orbit starting at $y_0$ $\beta$-shadows $(x_i)$.
  \end{lemma} 
  The logistic map is not hyperbolic...\\
  The following case study deals with the case of non-hyperbolic $f$. 
  As stated above, for this case we can not expect to find an orbit that shadows the numerical orbit forever. 
  Instead, there will be an orbit shadowing $(x_i)$ up to some point $n_0$ and then starting to diverge. 
  We want to invesitgate where this $n_0$ is.
  
   \cite{Hammel1987}
  \subsection{Implementation}
  \subsection{Evaluation}