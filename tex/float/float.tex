%!TEX root = ../../thesis.tex
\section{Floating Point Arithmetic}
The standard way to compute with real numbers on a computer is the use 
of \textbf{floating point arithmetic}. 

In general, the floating point representation of a real number has a fixed
number of bits (e.g. 32 or 64 bits on a modern computer architecture).
\begin{definition}\label{def: floating point number}
	A floating point representation with base $b$ and precision $p$
	is a string $\pm d_0 . d_1 \dots d_{p-1} \times b^e$ with $0 \leq d_i < b$.	
	It represents the number
	$$ \pm b^e \cdot \sum_{i=0}^{p-1} d_i\beta^{-i} $$ 

	$\pm d_0 . d_1 \dots d_{p-1}$ is called the Mantissa and $e$ the exponent.
	
	The smallest and largest exponent allowed are $e_min$ and $e_max$.
	A floating point number can be encoded in
	$$ \lceil log_2(e_{max}-e_{min}+1) \rceil + \lceil  p \cdot log_2 (b) \rceil + 1 $$
	bits.
\end{definition}
If a real number does not fit into the above representation, it has to be
rounded apropriately.
Also, when performing operations on floating point
numbers, the result will often not fit into the finite representation anymore,
and thus has to be rounded again. 

To make the floating point representation unique,
it is usually required that it is normalized, i.e. that $d_0 \neq 0$. 
\temp{Absolute and relative error.}

Floating point arithmetic usually works by
operating on
only a fixed number of digits for computation.
This makes computations fast, but can lead to large relative errors as in the following theorem: 
\begin{theorem}
	The relative error when computing $x-y$ for floating point numbers $x,y$ can be as large as $b-1$. 
\end{theorem}
Note: Guard digits
\subsection{The IEEE Standard}
	Floating point representations for binary base are standardized by IEEE 754 \cite{ieee}. 
	This standard is followed by almost all modern computers.
	The standard defines four different precisions: single, single-extended, double and double-extended.
	  \begin{table}
	  	\centering
	    \begin{tabular}{ | c || c | c | c | c | c | }
	    \hline
	    Type & $p$ & $e_{max}$ & $e_{min}$ & exponent width (bits) & format width (bits) \\ \hline \hline
	    Single & 24 & +127 & -126 & 8 & 32 \\ \hline
	    Single-Extended & 32 & +1023 & $\leq -1022$ & $\leq 11$ & 43 \\ \hline
	    Double & 53 & +1023 & -1022 & 11 & 64 \\ \hline
	    Double-Extended & 64 & > 16383 & $\leq -16382$ & 15 & 79 \\ \hline
	    \end{tabular}
	    \caption{Parameters for the different types in the IEEE 754 standard}\label{table:IEEE floating point}
	  \end{table}
	  
	The details can be seen in Table \ref{table:IEEE floating point}.
	The standard requires that the result of addition, subtraction, multiplication and division is exactly rounded, 
	i.e. the operation has to be performed exactly and then rounded.

	Floating point arithmetic is not necessarily associative, e.g.

	The standard also introduces special quantities $-0, +0, \inf, +\inf$ and NAN (not a number). 

	Rounding modes

	


