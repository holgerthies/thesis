%!TEX root = ../../thesis.tex
\section{Floating Point Arithmetic}
The standard way to compute with real numbers on a computer is the use 
of \textbf{floating point arithmetic}. 

In general, the floating point representation of a real number has a fixed number of bits (e.g. 32 or 64 bits on a modern computer architecture).
\begin{definition}\label{def: floating point number}
	A floating point representation with base $b$ and precision $p$
	is a string $\pm d_0 . d_1 \dots d_{p-1} \times b^l$ with $0 \leq d_i < b$.	
	It represents the number
	$$ \pm b^e \cdot \sum_{i=0}^{p-1} d_i\beta^{-i} $$ 
	The smallest and largest exponent allowed are $e_min$ and $e_max$.
	A floating point number can be encoded in
	$$ \lceil log_2(e_max-e_min+1) \rceil + \lceil  p \cdot log_2 (b) \rceil + 1 $$
	bits.
\end{definition}
If a real number does not fit into the above representation, it has to be rounded apropriately.
Also, when performing operations on floating point numbers, the result will often not fit into the finite representation anymore, 
and has to be rounded again.
To make the floating point representation unique, it is usually required that it is normalized, i.e. that $d_0 \neq 0$.
Absolute and relative error.

