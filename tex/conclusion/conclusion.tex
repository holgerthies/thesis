Two case studies have been presented, showing some of the possibilities and
limitations of exact real arithmetic.

The first case study was a reimplementation of a algorithm on floating point
numbers in exact real arithmetic.
The automatic precision control in \irram made the implementation much simpler.
Also, the \irram version makes it possible, to compute any point of the
shadowing orbit with arbitrary little error, instead of just giving bounds for
the position.

The second case study gave an example on how the theoretical results from
computable analysis can be applied to yield practical algorithms. 
The empirical evaluation of the running time, showed that the bounds found with
the help of real complexity theory correspond well with the implementation.
This can be seen as evidence, that the models used in computable analysis
map the reality quite well.

\section{Possible Future Work}
There are several possible extensions for both case studies.
In general, it would be interesting to reimplement the algorithms in some other
exact real arithmetic framework and compare the results.

Apart from that some possible extensions are listed below.

For the dynamical systems case studies the following is possible
\begin{enumerate}
  \item \textbf{Forward algorithm}
    An alternative to the algorithm implemented in the case study can be found in \cite{chow1991}.
    Instead of starting from the last point in the pseudo orbit and going backwards, this algorithm proceeds in forward direction.
    This algorithm could also be implemented in \irram and the results compared
    to the ones obtained in this thesis.
  \item \textbf{Multi-dimensional shadowing}
    Hammel, Yorke and Grebogi also briefly mention their results obtained for
    the two-dimensional case in their paper.
    They examined the Henon map
    \begin{eqnarray*}
      x_{n+1} &=& 1-ax_n^2+y_n \\
      y_{n+1} &=& -Jx_n
    \end{eqnarray*}
    where $J$ is the determinant of the Jacobian of the map.
    They found a similar shadowing result to the one mentioned in the case
    study for this map.
    In their implementation the true orbit is confined in a parallelogram to
    bound the position.
    \irram could be used to compute such an orbit exactly.
\end{enumerate}

The data-type for analytic functions could be extended in the following ways
\begin{enumerate}
  \item \textbf{Improvement of Running Time}
    The current implementation of the data type for functions analytic on
    $[0,1]$ uses iterated analytic continuation to evaluate the function at points where
    the given Taylor series is not valid. 
    This makes the representation simple, but the running time is exponential
    in the number of continuations.
    In practice, it is therefore not possible to have more than a few analytic
    continuations,
    Additional theoretical investigation should be done to find out, if it is
    possible to improve the running time, or if a high computational complexity
    is inherent to the problem.
  \item \textbf{Multi-dimensional functions}
    The implementation presented in this thesis only works with 1-dimensional
    functions. 
    For many applications, however, the analytic functions are
    more-dimensional.
    The data-type should be extended to work with multi-dimensional Taylor
    series.
  \item \textbf{Applications}
    It would also be interesting, to use the implementation to solve a
    practical problem involving analytic functions.
    Such problems could for example be found in the theory of Ordinary
    Differential Equations.
\end{enumerate}
