%!TEX root = ../../thesis.tex
\section{Introduction}
  \begin{definition}
    A \textbf{discrete dynamical system} is a triple $(\NN,X,\Phi)$ with $X$ a non empty set (the state space) and an operation $\Phi : \NN \times X \to X$, so that
    $\Phi(0,x) = x$ and $\Phi(n,\Phi(m,x)) = \Phi(n+m, x)$. 
    Thus, a discrete dynamical system is the model of a system, that evolves over discrete time steps.
  \end{definition}
  The simplest case of a discrete dynamical system, is when the state space is 1-dimensional (say $X \subseteq \RR$) and 
  the transition function only depends on the previous value, i.e. $\Phi(1,x) = f(x)$ for some $f : \RR \to \RR$. 

  In that case the dynamical system can be written by the recurrence relation $x_{n+1} = f(x_n)$ with initial condition $x_0 \in \RR$.
  A chaotic system is a dynamical system, that is highly sensitive to initial conditions.
  A consequence of that is that even relatively small numerical errors in the computation grow exponentially fast.
  A prototypical example of a chaotic system is the logistic map:
  \begin{definition}\label{def:logmap}
    The \textbf{logistic map} is given by the recurrence relation 
    $ x_{n+1} = ax_n(1-x_n) \text{ with } a, x_0 \in \RR$.
  \end{definition} 
  \begin{table}
    \begin{tabular}{ | c | c | }
    \hline
    $a$ & behaviour \\ \hline
    $[0,1]$ & Stabilizes at $0$ \\ \hline
    $(1,3]$ & Stabilizes at $\frac{a-1}{r}$ \\ \hline 
    $(3, \approx 3.544]$ & oscillates between $2$, $4$, $8$, $\dots$ values. \\ \hline 
    $(\approx 3.544, 4]$ & \begin{tabular}{l}almost all inital points no oscillation with finite period \footnote{This is true for most points in this interval. There are however isolated ranges of $a$ for which the oscillation period is finite.}  \\ Small changes in the initial point yield large differences over the iterations. \end{tabular} \\ \hline
    $(4, \infty)$ & The values eventually leave the interval [0,1] and diverge for almost all initial points \\ 
    \hline
    \end{tabular}
    \caption{behaviour of iterating the logistic map for different values of the parameter $a$}\label{table:logmapbehaviour}
  \end{table}
  \begin{figure}[H]
    \centering
    \includegraphics[width=0.5\textwidth]{img/dynamic_systems/logmap1}
    \caption{100 Iterations of the logistic map with $a=3.8$ and $x_0 = 0.4$ computed with 10 decimals precision and exactly.}
    \label{fig:logmaperror1}  
  \end{figure}
    \begin{figure}[H]
    \centering
    \includegraphics[width=0.5\textwidth]{img/dynamic_systems/logmap2}
    \caption{Maximal precision iRRAM uses to compute iterations of the logistic map and output the points with 10 decimal digits.}
    \label{fig:logmapprec}
  \end{figure}
  The logistic map behaves differently for different values of the parameter $a$ as can be seen in Table \ref{table:logmapbehaviour}. 
  The for this thesis interesting case is for $a \in (3.544, 4)$ where the iteration of the map leads to chaotic behaviour.

  Since the \irram framework can be used to compute iterations of the logistic map exactly, it is very well suited to invesitgate this chaotic behaviour \footnote{what could also be shown in a contest in 2000 \cite{competition:2001}}. 

  Figure \ref{fig:logmaperror1} shows how sensitive the map is to small errors that occur because of finite precision computations. 
  When using 10 significant digits for the computation, it can be seen that after around 50 iterations the finite precision version diverges from the \irram version and then behaves completely differently. 

  \irram can also be used to approximate the complexity in terms of needed computation precision. Since \irram automatically increases the internal precision until it suffices to output the result with the demanded number of digits, the internal precision at the end of the computation gives a measure of this complexity. As can be seen in Figure \ref{fig:logmapprec}, the needed internal precision grows extremely quickly with the number of iterations.
  To compute 10 significant digits of the $20000$-th iterate, \irram interally computes already with precision nearly $2^{-50000}$, which is of course far from any standard floating point datatype.   
  \subsection{The Shadowing Lemma}
    The term pseudo-orbit is used to describe numerically generated noisy orbits. 
    \begin{definition}\label{def:pseudoorbit}
      A sequence $(x_i)$ is called an \textbf{$\alpha$-pseudo-orbit} for a map $f$ if
      $ \| x_{i+1} - f(x_i) \| < \alpha $  
    \end{definition}
    One can think of a pseudo orbit as a numerically computed orbit, where small rounding errors can occur in every evaluation of $f$.
    \begin{definition}\label{def:shadowing}
      A real orbit $(y_i)$ \textbf{$\beta$-shadows} the pseudo-orbit $(x_i)$ if 
      $\| x_i - y_i \| < \beta$.  
    \end{definition}
    \begin{definition}
      A dynamic system is called \textbf{uniformly hyperbolic} if ...
    \end{definition}
    For systems that are uniformly hyperbolic Anosov and Bowen could show the following result \cite{anosov1967} \cite{Bowen1975} \cite{Hasselblatt:2008}:
    \begin{theorem}[Shadowing Lemma]
     For all $\beta > 0$ there exists an $\alpha > 0$ so that for every $\alpha$-pseudo-orbit $(x_i)$ there is a point $y_0$ so that the real orbit starting at $y_0$ $\beta$-shadows $(x_i)$.
    \end{theorem} 
    The logistic map is not hyperbolic...\\
    So for non-hyperbolic $f$, it can not be expected, that there is a real orbit that stays close to the pseudo orbit forever.
    Instead, there will be an orbit staying close to $(x_i)$ for some time, say up to some point $n_0$, and then starting to diverge. 
    The goal of this the case study is to investigate such orbits, in particular to find out how long there is an orbit staying close to the pseudo orbit,
    The basis for this case study is a paper by Hammel, Yorke and Grebogi from 1987 \cite{Hammel1987}.
    The paper deals with the question, how long numerical orbits of the logitstic map can be shadowed by true orbits.
    The authors show that for $a = 3.8$ and $x_0 = 0.4$, there is a true orbit $(y_n)$ of the logistic map, so that $\| x_n - y_n \| < 10^{-7}$ for $n \leq 10^7$.
    Their proof method is to compute bounds on the points of the true orbit by using a form of interval arithmetic and letting a computer do the computations. 
    All their computations are done on a Cray X-MP supercomputer. 
    For the case study, first their algorithm was implemented on a modern computer in the \cc programming language, both using fixed precision floating point numbers and \irram. 
    As a second step \irram was used to compute the shadowing orbit exactly.
  \subsection{Backward Algorithm}
    The goal of the algorithm is to compute a shadowing orbit $(y_n)$ of size $N$ for a pseudo orbit $(x_n)$ of the logistic map $f$.
    Instead of computing the shadowing orbit from the first point, the algorithm computes the inverse orbit by starting with the last point and then iteratively computing the predecessor. 
    Since the logistic map is not injective, there is not a unique choice for this predecessor. 
    For some point $f(x)$ there are normally two possible values for the inverse given by $f^{-1}_{1,2}(x) = 0.5 \pm \sqrt{0.25 + \frac{x}{a}} $.
    An orbit can be computed by setting $y_N = x_N$ and then iteratively applying one of the inverse map $y_{n-1} = f^{-1}(y_n)$.   To stay close to the pseudo orbit, in every step the point on the same side of $0.5$ as $x_n$ is chosen (the index of $f^{-1}$ will from now on be omitted). 
    This works since the inverse function on $(0,1)$ is a contraction, thus
    $$ \| x_{n-1} - y_{n-1} \| = \| x_{n-1} - f^{-1}(y_n) \| \leq \| f^{-1}(x_n) - f^{-1}{y_n} \| + \| f^{-1}(x_n) - x_{n-1} \| \leq \| x_n - y_n \| + \delta $$   
    Since the original work had to do all the computations using floating point arithmetics, the inverse could not be computed exactly.
    Instead of the points $y_n$, a sequence of intervals $(I_n)$ that bound the location of $y_n$ is computed, i.e. so that $y_n \in I_n \text{ for } 0 \leq n \leq N$.
    The procedure is started with $I_N := [x_n, x_n]$ and then $I_{n-1}$ is selected so that $I_n \subseteq f(I_{n-1})$ holds and $I_n$ is chosen as small as possible. 
    Then the maximal distance between all points in $I_n$ and $x_n$ is computed.
    This gives a shadowing bound.
    \subsection{Forward Algorithm}
    An alternative to the algorithm descriped above can be found in \cite{chow1991}.
    Instead of starting from the last point in the pseudo orbit and going backwards, this algorithm proceeds in forward direction.
    They define the two quantities
    \begin{eqnarray*}
    \sigma & = & \sup^N_{n=0} \sum_{m=n}^N  \prod_{j=n}^{m} | Df(y_j)^{-1} |  \\
    \tau & = & \sup^N_{n=0} | \sum_{m=n}^N (y_{m+1}-f(y_m)\prod_{j=n}^{m} Df(y_j)^{-1}  |
    \end{eqnarray*}
    and show the following theorem:
    \begin{theorem}
    Let $f: [0,1] \to [0,1] \in C^2$ and $M = sup \{|D^2f(x)| \,:\, x \in [0,1] \}$.
    For a pseudo orbit $(y_n)_{n=0}^{N+1}$ with $2M\sigma\tau \leq 1$, there is an exact orbit $(x_n)_{n=0}^N$  with 
    $$sup_{n=0}^N |x_n - y_n| \in [\frac{tau}{1+0.5(1+\sqrt{1-2M\sigma \tau})}, \frac{2\tau}{1+\sqrt {1-2M\sigma \tau}}]$$
    \end{theorem}
    Then they derive the following esitmations for $\sigma$ and $\tau$:
    The procedure to determine how closely a given pseudo orbit $(y_n)$ is shadowed by a true orbit is then given as follows:
    \begin{enumerate}
      \item Find upper bounds for $M$, $\delta$ and $\Delta$.
      \item Compute 
      \begin{eqnarray*}
      \mu_p &=& \sup_{n=0}^{N-p} \prod_{m=n}^{n+p} |Df(y_m)^-1| \\
      \sigma_p &=& \sup_{n=0}{N} \sum_{m=n}^{min(n+p,N)} \prod_{j=n}^{m} |Df(y_j)^-1| \\
      \tau_p &=& \sup_{n=0}{N} \sum_{m=n}^{min(n+p,N)} \prod_{j=n}^{m} |Df(y_j)^-1| \\
      \end{eqnarray*}
    \end{enumerate}
\section{Implementation}
  The first part of the case study was to simulate the computations that were made on the Cray X-MP supercomputer on a modern computer. 
  The cray double precision data type used for the computations in the paper has machine epsilon $\varepsilon_\mu = 2^{-95}$.
  To compute the intervals $I_n$ from $I_{n+1}$ the inverse of the interval is computed as described above (using floating point arithmetic).
  Then as long as $I_{n+1} \not \subseteq f(I_n)$, $I_n$ is enlarged on both sides by the minimal possible quantity $\varepsilon_\mu$. 
  Since also the computation of $f(I_n)$ is done using floating point arithmetics, the condition might still not be fulfilled.
  This can, however, be compensated by further enlargening the interval on both sides by the constant $10^{-25}$.
  Since the precision of the clay double data type differs from the IEEE double precision, it was not possible using \cc's built in data types for simulating the algorithm.
  Instead, the boost multiprecision library \cite{boostmultiprecision} was used. 
  The library provides data types that replace the native \cc floating point types, but with a user defined precision. 
  For the interval arithmetic an interval class was implemented, having the following methods:\\
  CLASS DIAGRAM \\
  The same algorithm was also implemented in a different way. 
  Instead of using intervals with fixed precision floating point end points, \irram's \code{INTERVAL} type can be used. 
  The data type provides interval arithmetic for intervals with real numbered end points.
  Then, the inverse of an interval can be computed exactly.
  
  Instead of computing only an upper bound to the shadowing bound \irram can also be used to compute the exact shadowing orbit. 
  To do that, one has to compute the pseudo orbit by computing an orbit using floating point arithmetic of a certain fixed precision.
  Then, starting with the last point of the pseudo orbit the inverses can be computed exactly until the initial point is reached.
  Then the distance of this real orbit and the pseudo orbit can be computed and the maximum of all those distances taken to find the shadowing distance.
  The \irram code to compute the exact orbit is extremely simple.
  For finding the shadowing bound, however, one has to be a little more careful, since the maximum of two real numbers is not computable and can lead to infinite recursion in \irram. 
  Since there only had to be found a qualitative behavior of the shadowing bound depending on the various parameters instead of finding the exact distance as a real number, it does not really matter if the maximum distance was chosen or only a number that is close enough to the maximum, so that it does not matter for the analysis. 
  \irram's support for multivalued functions yields an easy solution to this problem: 
  The maximum computation may be wrong, when two numbers are very close to each other (their distance is orders of magnitude lower than the result we expect). 
  This program will always terminate and give a very accurate approximation to the shadowing distance.  
\section{Evaluation}
  \subsection{Breakdown points}
  Since the inverse of the logistic map is only defined for points $x < \frac{a}{4}$, the above procedure fails, if at some point $i$ $x_i > \frac{a}{4}$. The first $N$ for that such an $i$ exists, is called the breakdown point.
  The breakdown point depends on the precision that are used to compute the pseudo orbit. 
  \subsection{Shadowing bound and computation precision}
  The original work also deals with the question, how the computation precision is related to the shadowing bound. 
  Since the computation precision on the Clay X-MP was determined by the available floating point data types, it was not changed. Instead, the noise was artificially increased by setting $x_{n+1} = ax_n(a-x_n)+\delta_p \cos (\Theta)$ with $\Theta = n (mod 997)$ and then the shadowing distance in relation to $\delta_p$ measured.
  However, using the boost multiprecision library's \code{cpp\_bin\_float} data type, an arbitrary number of binary digits can be used for the precision.

  \begin{figure}[H]
    \centering
    \begin{subfigure}{.45\textwidth}
      \centering
      \includegraphics[width=1.0\textwidth]{img/dynamic_systems/dist_prec_N_1000}
      \caption{$10^3$ iterations}
    \end{subfigure}
     \begin{subfigure}{.45\textwidth}
      \centering
      \includegraphics[width=1.0\textwidth]{img/dynamic_systems/dist_prec_N_10000000}
      \caption{$10^7$ iterations}
    \end{subfigure}   
    \caption{Shadowing distance for different parameters $a$ and $p_0$ of the logistic map. The distance is bound by $\frac1{\sqrt \delta}$ where $\delta = 2^{-n}$ is the precision.}
    \label{fig:shadowingdistance1}
  \end{figure}

  The shadowing distance was computed for several different parameters $a$ and starting points $p_0$. THe number of iterations $N$ were also varied. 
  The result can be seen in Figure \ref{fig:shadowingdistance1}. A complete table of all the starting values and parameters can be found in the Appendix. Hammel, Yorke and Grebogi conjecture that for $2M$-digit accuracy, that the orbit stays close to $10^{-M}$ for around $10^M$ iterates. As can be seen in the graph, for up to $10^7$ iterates the conjecture holds for all the tested values.
  The results obtained by the \irram version and the version working with fixed precision interval arithmetic were nearly identical. 


  Shadowing bound for different parameters
  Breaking point for different precisions
  Real orbit?
