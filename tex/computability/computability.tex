%!TEX root = ../../thesis.tex
\section{Real Computability Theory}
\subsection{Classical Computability Theory}
 Since in real computability theory many aspects of classical computability theory are extended, 
 a very brief overview is given in the following section.
 Turing-Machines are great...
 \begin{definition}
 	A possibly partial function $f:\subseteq \Sigma^* \to \Sigma^*$ is called \textbf{computable} if there exists 
 	a Turing-Machine, so that for all $x \in dom(f)$ the machine terminates after finitely many steps with $f(x)$ on its 
 	tape and for $x \not \in dom(f)$ the machine does not terminate.
 \end{definition}

 Church-Turing-Thesis...
 \begin{definition}
 	A set $A \subseteq \Sigma^*$ is called \textbf{decidable}, if its characteristic function is computable. 
 \end{definition}
 \begin{definition}
 	A set $A \subseteq \Sigma^*$ is called \textbf{recursively enumerable} (r.e.) or \textbf{computably enumerable} (c.e.) if 
 	it is empty or if $A$ is the domain of a computable function.   
 \end{definition}
\subsection{Computability of real numbers}
The previous section showed how to define computability over finite alphabets $\Sigma^* \to \Sigma^*$. 
That is enough to define computability for finite structures. The following section defines how to extend 
the classical notion to uncountable objects such as real or complex numbers, functions or infinite sequences.
There are several non equivalent ways, to define computability on such objects. 
In contrast to the classical case, where the definition given in the previous section is widely accepted, there is no 
generally accepted model for real complexity theory.

The model used in this thesis is the so called \textbf{Type 2 Theory of Effectivity} (TTE). 
Because... 

This section gives an overview of the framework used to define computability on real numbers and is therefore 
a little more general then needed in the rest of the thesis, that focuses more on implementations of subsets of this framework.
However,...

The general definition is by Type-2 Turing Machines...
From now on the term Turing-Machine is used both for classical and type 2 machines, when it is clear by context which model is meant.
\begin{definition}\label{def:computability_ttt}
A function $F:\subseteq \Sigma^\omega \to \Sigma^\omega$ is called computable if there is a Turing-Machine  
that for all infinite strings $\sigma in dom(F)$ writes the infinite string $F(\sigma)$ on its output tape. 
For $\sigma \not \in dom(f)$ the machine writes only finitely many symbols on the output tape.  
\end{definition}
To talk about computability over some set, the notion of encoding this set to $\sigma^\omega$ has to be formalized.
\begin{definition}\label{def:representation}
	A \textbf{representation} of a set $X$ is a partial surjective mapping $\alpha: \sigma^\omega \to X$. \\
	$\bar \sigma \in \alpha^{-1}(\sigma)$ is called an \textbf{$\alpha$-name} of $\sigma$. \\
	$x \in X$ is \textbf{$\alpha$-computable} if it has a decidable $\alpha$-name.
\end{definition}

Of course, the definition of representations is open wide and can lead to many different more or less useful definitions of computability.
Some possible representations for real numbers are as follows
\begin{enumerate}
\item A $\rho_{10}$-name of $x$ is the usual decimal expansion of $x$.
\item A $\rho$-name of $x \in \RR$ is a sequence $a_n \in \ZZ$ s.t. $| x - a_n | \leq 2^{-n}$
\item  A $\rho_C$-name of $x \in \RR$ consists of two sequences rational $(q_n)_{n \in \NN}$ and $(\varepsilon_n)_{n \in \NN}$, so that 
$| x_n - q_n | < \varepsilon_n$ and $\lim_{n \to \infty} \varepsilon_n = 0$  
\item $\rho_<$-name, $\rho_>$-name
\item $\rho_n$-name 
\end{enumerate}
\begin{theorem}
The following are equivalent
\item $x \in \RR$ is computable in the sense of Definiton 
\item $x \in \RR$ is $\rho_{10}$ computable
\item $x \in \RR$ is $\rho$-computable
\end{theorem}
\begin{example}
Specker-Sequence
\begin{definition}\label{def:representation_composition}
Composition of representations
\end{definition}
\end{example}
This thesis does not deal with the problem of computing single real numbers, but rather with computing real functions and real functionals.
For a function $f:\subseteq \RR \to \RR$ to be computable means, that is it possible to compute $f(x)$ arbitrarily good. 
Also a machine can not read $x$ with infinite precision, but it can ask to get $x$ as exact as it needs it to compute the output.
\begin{definition}\label{def:computability_oracle_tm}
\end{definition}
This concept can be generalized by the following definiton  
\begin{definition}\label{def:computability_function_representation}
	A function $f: \subseteq X \to Y$ is called \textbf{$(\alpha, \beta)$}-computable, 
	if there exists a computable function $F:\subseteq \Sigma^\omega \to \Sigma^\omega$ such that 
	$\beta(F(\sigma)) \in f(\alpha(\sigma))$ for all $\sigma \in dom(f \circ \alpha) $.  
\end{definition}
The Diagram in ... shows computability.
Examples...
\begin{theorem}
Multiplication is not $(\rho_{10}, \rho_{10})$-computable.
\end{theorem}
From now on the term computable will be used to describe computability w.r.t. the Cauchy representation.
Then all of the following functions are computable
\begin{enumerate}
\item Arithmetical operations $+,-,x,/ : \subseteq \RR^2 \to \RR$
\item The absolute value function
\item The minimum and maximum functions
\item constant functions with computable constant
\item Projections $\RR^k \to \RR$ 
\item polynomials with computable coefficients
\item $exp, sin, cos$
\item The square-root function and the logarithm function
\end{enumerate}
\begin{theorem}
	Computable functions are continuous...
	\begin{proof}
		Maybe I will do the proof later...
	\end{proof}
\end{theorem}

\begin{theorem}
Computability is preserved under function composition, i.e.
For sets $X,Y,Z$ with representations $\delta_X, \delta_Y, \delta_Z$, 
$f:\subseteq X \to Y$ $(\delta_X, \delta_Y)$-computable and $g:\subseteq Y \to Z$ $(\delta_Y, \delta_Z)$-computable,
$g \circ f$ is $(\delta_X, \delta_Z)$-computable.
\end{theorem}
\begin{definition}
A multi-valued function $f: \subseteq X \rightrightarrows Y$ is just an other name for a relation $f \subseteq X \times Y$.
A multi-valued function is $(\rho_X, \rho_Y)$ computable, if there is a a comnputable (single valued) function 
$F: \subseteq \Sigma^\omega \to \Sigma^\omega$ such that for all $\sigma \in dom(f \circ \rho_X)$, $\rho_Y(F(\sigma)) \in f(\rho_X(\sigma))$. 
\end{definition}
\subsection{Computability of real operators and functionals}
A real operator maps functions $\RR \to \RR$ to functions $\RR \to \RR$and a functional maps functions $\RR \to \RR$ to real numbers $\RR$.
For that, a representation for the space to work on is needed.
Continuous functions on a compact subset $X \subseteq R^d$ can be uniformly approximated by polynomials arbitrarily close.
A possible representation for real valued functions is thus given by the following definition 
\begin{definition}
A $[\rho^d \to \rho]$-name of a function $f \in C([0,1]^d, \\R)$ is given by a sequence $P_n \in \\D[x1, \dots, x_d]$ of polnomials (i.e. degree and list of coefficients), such that $\vert f - P_n \vert_\infty < 2^{-n}$   
\begin{theorem}[name?]
The integration operator 
$$I: C[0,1] \ to C[0,1], f \to (x \to \int_0^x f(t) dt$$   
is computable.
\end{theorem}
\begin{theorem}[Myhill 1971]
There is a computable function $f: [0,1] \to \R$ with continuous but uncomputable derivative. 
\end{theorem}
\begin{theorem}
The operator 
$$ D: C^1[0,1] \to C[0,1], f \to f'$$
is computable.
\end{theorem}
\end{definition}
\subsection{Uniformity and Non-Uniformity}
When talking about computability one has to distinguish between two types of computability, \textbf{uniform} and \textbf{non-uniform}.
FOr non-uniform computability it suffices, that for every input, there is an algorithm that computes the output. 
The algorithm may however depend on the input in a non-computable way.
In contrast, a problem is uniformly computable only if there is one algorithm, that computes the output for every valid input. 
Intermediate Value Theorem
\subsection{Other models for comoutable reals}
Markov Computability \\ 
Sequential Computabilty \\
Computable invariance \\
BSS-model \\