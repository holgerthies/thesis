%!TEX root = ../../thesis.tex
\section{Real Computability Theory}\label{sec:real computability}
\subsection{Classical Computability Theory}
 Since in real computability theory many aspects of classical computability theory are extended, 
 a very brief overview is given in the following section. 
 A detailed introduction can for example be found in \cite{computability}

 To define computability the Turing-Machine model is used.
 The Turing-Machine was invented by Alan Turing in 1936 \cite{Turing} and can
 be seen as simplified mathematical model for a computer.

 The machine constists of an infinite tape that is divided into cells. 
 Each cell contains exactly one symbol from a predefined finite alphabet
 $\Sigma$.
 It further consists of a head that is always positioned on top of one cell 
 and can in one step read and write the content of the cell and then move one
 cell left or right on the tape.

 The machine is always in one of finitely many states and has a finite
 instruction table containing instructions of the form when in state $q$ and
 reading symbol $s$, write $s'$ and move head to the left (or to the right).

 A formal definition can for example be found in \cite{Hopmann}.
 
 \begin{definition}
 	A possibly partial function $f:\subseteq \Sigma^* \to \Sigma^*$ is called \textbf{computable} if there exists 
 	a Turing-Machine, so that for all $x \in dom(f)$ the machine terminates after finitely many steps with $f(x)$ on its 
 	tape and for $x \not \in dom(f)$ the machine does not terminate.
 \end{definition}

 Even tough the Turing-Machine model is quite simple, it can be used to
 simulate every computer algorithm.
 The widely believed \textbf{Church-Turing thesis} even states, that anything
 that is computable in an informal sense, can be computed with a
 Turing-machine.

 Aoart from the computability notion for functions, there are also the
following notions on sets:
 \begin{definition}
 	A set $A \subseteq \Sigma^*$ is called \textbf{decidable}, if its characteristic function is computable. 
 \end{definition}
 \begin{definition}
 	A set $A \subseteq \Sigma^*$ is called \textbf{recursively enumerable} (r.e.) or \textbf{computably enumerable} (c.e.) if 
 	it is empty or if $A$ is the domain of a computable function.   
 \end{definition}
\subsection{Computability of real numbers}
The previous section showed how to define computability for functions over finite alphabets $\Sigma^* \to \Sigma^*$. 
That is enough to define computability for finite structures, but does not
suffice to make any statements on real numbers.

The following section defines how to extend the classical notion to uncountable
objects such as real or complex numbers, functions or infinite sequences.

There are several non equivalent ways, to define computability on such objects. 
In contrast to the classical case, where the definition using Turing-machines is widely accepted, there is no 
generally accepted model for real complexity theory.

One possible model is the so called \textbf{Type 2 Theory of Effectivity}
(TTE). \cite{Weihrauch} 
This model aims to realistically model what is possible to compute with a
computer and is useful in the analysis of the algorithms that will follow
later.
It is therefore the sole model used in this thesis. 

TTE extends the Turing-Machine model to so called type-2 Turing-machines, that
can work on infinite strings $s \in \Sigma^w$. 
\begin{definition}
A type-2 Turing-machine is a multi-tape Turing-machine with two special tapes,
an \textbf{input tape} and an \textbf{output tape}, and at least one working
tape. 
The input tape is read-only, i.e. it is not possible to change a cell on the
tape. Further, the head on both input and output tape can not be moved to the
left. In particular it is not possible to change a symbol that has been written
on the output tape once. 
\end{defintion}
From now on the term Turing-Machine is used both for classical and type 2
machines, when it is obvious by context which model is meant.
\begin{definition}\label{def:computability_ttt}
A function $F:\subseteq \Sigma^\omega \to \Sigma^\omega$ is called computable if there is a Turing-Machine  
that for each infinite strings $\sigma in dom(F)$ on its input tape, writes the infinite string $F(\sigma)$ on its output tape. 
For $\sigma \not \in dom(f)$ the machine writes only finitely many symbols on the output tape.  
\end{definition}
To talk about computability over some arbitrary set, it has to be encoded to
$\sigma^\omega$. 
This encoding is formalized in the term representation.
\begin{definition}\label{def:representation}
	A \textbf{representation} of a set $X$ is a partial surjective mapping $\alpha: \sigma^\omega \to X$. \\
	$\bar \sigma \in \alpha^{-1}(\sigma)$ is called an \textbf{$\alpha$-name} of $\sigma$. \\
	$x \in X$ is \textbf{$\alpha$-computable} if it has a decidable $\alpha$-name.
\end{definition}

In contrast to the countable case, where a canonical encoding is obvious in most
cases, finding a good representation is more challenging in the uncountable
case.
Different representations can lead to a different computability notion.
Thus it is important to find a representation that leads to a useful and
realistic notion of computability.

Some possible representations for real numbers are as follows
\begin{enumerate}
\item A $\rho_{10}$-name of $x$ is the usual decimal expansion of $x$.
\item A $\rho$-name of $x \in \RR$ is a sequence $a_n \in \ZZ$ s.t. $| x - a_n | \leq 2^{-n}$
\item  A $\rho_C$-name of $x \in \RR$ consists of two sequences rational $(q_n)_{n \in \NN}$ and $(\varepsilon_n)_{n \in \NN}$, so that 
$| x_n - q_n | < \varepsilon_n$ and $\lim_{n \to \infty} \varepsilon_n = 0$  
\item $\rho_<$-name, $\rho_>$-name
\item $\rho_n$-name 
\end{enumerate}
Some of the above notions lead to an equivalent definition of a computable real
numbers, but others do not.

In particular the following are equivalent
\begin{theorem}
\begin{enumerate}
  \item $x \in \RR$ is computable in the sense of Definiton 
  \item $x \in \RR$ is $\rho_{10}$ computable
  \item $x \in \RR$ is $\rho$-computable
\end{enumerate}
\end{theorem}
A real number is called \textbf{computable} if it is computable in the sense of one of
those defintiions.

It can easily be seen that there must be non=computable reals, since the number
of reals is uncountable, while the number of type-2 Turing-machines is
countable.

The following gives an explicit construction for a non-computable real number
\begin{example}
Let $A \subseteq \NN$ be any recursively enumerable, but not decidable subset
of the natural numbers.
Define the real number $x$ by
$$ x := \sum_{n \in A} 2^{-n}. $$
Note, that this is well defined since the series of partial sums is strictly
increasing and bounded by $2$.

However, $x$ can not be computable since otherwise it would yield a decision
procedure for $A$. 
\end{example}
A sequence as the above, that is computable, monotonic, bounded and consists
only of rational numbers, but has a non-computable supremum is called \textbf{Specker-sequence}.

\begin{definition}\label{def:representation_composition}
Composition of representations
\end{definition}
This thesis does not deal with the problem of computing single real numbers, but rather with computing real functions and real functionals.
For a function $f:\subseteq \RR \to \RR$ to be computable means, that is it possible to compute $f(x)$ arbitrarily good. 
Also a machine can not read $x$ with infinite precision, but it can ask to get $x$ as exact as it needs it to compute the output.
\begin{definition}\label{def:computability_oracle_tm}
\end{definition}
This concept can be generalized by the following definiton  
\begin{definition}\label{def:computability_function_representation}
	A function $f: \subseteq X \to Y$ is called \textbf{$(\alpha, \beta)$}-computable, 
	if there exists a computable function $F:\subseteq \Sigma^\omega \to \Sigma^\omega$ such that 
	$\beta(F(\sigma)) \in f(\alpha(\sigma))$ for all $\sigma \in dom(f \circ \alpha) $.  
\end{definition}
The Diagram in ... shows computability.
Examples...
\begin{theorem}
Multiplication is not $(\rho_{10}, \rho_{10})$-computable.
\end{theorem}
From now on the term computable will be used to describe computability w.r.t. the Cauchy representation.
Then all of the following functions are computable
\begin{enumerate}
\item Arithmetical operations $+,-,x,/ : \subseteq \RR^2 \to \RR$
\item The absolute value function
\item The minimum and maximum functions
\item constant functions with computable constant
\item Projections $\RR^k \to \RR$ 
\item polynomials with computable coefficients
\item $exp, sin, cos$
\item The square-root function and the logarithm function
\end{enumerate}
\begin{theorem}
	Computable functions are continuous...
\end{theorem}

\begin{theorem}
Computability is preserved under function composition, i.e.
For sets $X,Y,Z$ with representations $\delta_X, \delta_Y, \delta_Z$, 
$f:\subseteq X \to Y$ $(\delta_X, \delta_Y)$-computable and $g:\subseteq Y \to Z$ $(\delta_Y, \delta_Z)$-computable,
$g \circ f$ is $(\delta_X, \delta_Z)$-computable.
\end{theorem}
\begin{definition}
A multi-valued function $f: \subseteq X \rightrightarrows Y$ is just an other name for a relation $f \subseteq X \times Y$.
A multi-valued function is $(\rho_X, \rho_Y)$ computable, if there is a a comnputable (single valued) function 
$F: \subseteq \Sigma^\omega \to \Sigma^\omega$ such that for all $\sigma \in dom(f \circ \rho_X)$, $\rho_Y(F(\sigma)) \in f(\rho_X(\sigma))$. 
\end{definition}
\subsection{Computability of real operators and functionals}
A real operator maps functions $\RR \to \RR$ to functions $\RR \to \RR$and a functional maps functions $\RR \to \RR$ to real numbers $\RR$.
For that, a representation for the space to work on is needed.
Continuous functions on a compact subset $X \subseteq R^d$ can be uniformly approximated by polynomials arbitrarily close.
A possible representation for real valued functions is thus given by the following definition 
\begin{definition}
A $[\rho^d \to \rho]$-name of a function $f \in C([0,1]^d, \\R)$ is given by a sequence $P_n \in \\D[x1, \dots, x_d]$ of polnomials (i.e. degree and list of coefficients), such that $\vert f - P_n \vert_\infty < 2^{-n}$   
\begin{theorem}[name?]
The integration operator 
$$I: C[0,1] \ to C[0,1], f \to (x \to \int_0^x f(t) dt$$   
is computable.
\end{theorem}
\begin{theorem}[Myhill 1971]
There is a computable function $f: [0,1] \to \R$ with continuous but uncomputable derivative. 
\end{theorem}
\begin{theorem}
The operator 
$$ D: C^1[0,1] \to C[0,1], f \to f'$$
is computable.
\end{theorem}
\end{definition}
\subsection{Uniformity and Non-Uniformity}
When talking about computability one has to distinguish between two types of computability, \textbf{uniform} and \textbf{non-uniform}.
FOr non-uniform computability it suffices, that for every input, there is an algorithm that computes the output. 
The algorithm may however depend on the input in a non-computable way.
In contrast, a problem is uniformly computable only if there is one algorithm, that computes the output for every valid input. 
Intermediate Value Theorem
\subsection{Other models for comoutable reals}
Markov Computability \\ 
Sequential Computabilty \\
Computable invariance \\
BSS-model \\
