%!TEX root = ../../thesis.tex
\section{Real Computability Theory}
\subsection{Classical Computability Theory}
 Since in real computability theory many aspects of classical computability theory are extended, 
 a very brief overview is given in the following section.
 Turing-Machines are great...
\subsection{Computability of real numbers}
The previous section showed how to define computability over finite alphabets $\Sigma^* \to \Sigma^*$. 
That is enough to define computability for finite structures. The following section defines how to extend 
the classical notion to uncountable objects such as real or complex numbers, functions or infinite sequences.
There are several non equivalent ways, to define computability on such objects. 
In contrast to the classical case, where the definition given in the previous section is widely accepted, there is no 
generally accepted model for real complexity theory.
The model used in this thesis is the so called \textbf{Type 2 Theory of Effectivity} (TTE). 
Because... 
The general definition is by Type-2 Turing Machines...

