%!TEX root = ../../thesis.tex
\section{A Data Structure for Analytic Functions}
	Acknowledgment: Florian, Akitoshi
	\subsection{Introduction}
		As seen in Chapter \ref{section:real_complexity} many important operators 
		on real functions have been shown to be computationally hard.
		On the other hand, numerical scientists can quickly compute maxima or anti-derivatives of functions.
		To analyze this mismatch one should ask the follwing questions
		\begin{enumerate}
			\item What subset of real functions is usually considered in numerical analysis?
			\item Are there implicit assumptions made about the functions, that allow faster computations? 
		\end{enumerate}
		One class of functions that has been considered in computable analysis is the analytic functions.
		Although not at all sufficent as the class of functions considered interesting for numerical analysis, 
		many computations become tractable when one restricts the attention to analytic functions.
		\begin{definition}
			For $z \in \CC$ and $r \in \RR, r > 0$, let $B(z,r) := \{ x \in \CC \,\, | x - z | < r\}$. \\
			A function $f: D \to \CC$ with $D \subseteq CC$ is called \textbf{analytic} if for every $x_0 \in D$ 
			the function given by the Taylor-series around $x_0$ converges to $f(x)$ for every $x$ in a neighborhood of 
			$x_0$. 
			That is, for every $x_0 \in \D$ there is a $\varepsilon > 0$, such that for all $x \in B(x_0, \varepsilon)$ and for 
			$$ T(x) := \sum_{n=0}^{\infty} \frac{f^{(n)}(x_0)}{n!}(x-x_0)^n  \text{, } T(x) \rightarrow f(x)$$.
			The set of functions analytic on $D$ is denoted as $C^\infty(D)$. \\
			The coefficients of the Taylor expansion is denoted by the series $(a_n)_{n \in \NN}$, i.e. 
			$$ a_n = \frac{f^{(n)}(x_0)}{n!}. $$    
			For the sake of simplicity, it will w.l.o.g. be assumed that $x_0 = 0$  and $D = [0,1]$ if not stated otherwise.
		\end{definition}
		Note: The above definition can be easily generalized to higher dimension. \\
		Some properties of analytic functions...
		The complexity of analytic functions is much better then for the general case.
		\begin{theorem}[Pour-El, Richards, Ko, Friedman, Müller] 
			The following are equivalent
			\begin{enumerate}
				\item $f$ is computable 
				\item The series $(a_n)_{n \in \NN}$ is computable. 
 			\end{enumerate}
 			The equivalence also holds, when computable is replaced by polynomial time computable.
		\end{theorem}
		From that it follows
		\begin{corollary}
			If $f$ is polynomial time computable the following functions are
			\begin{enumerate}
				\item $I: [0,1] \to \RR$, $I(x) = \int_0^x f(t) dt$
				\item $D: [0,1] \to \RR$, $D(x) = f'(x)$ 
			\end{enumerate}
			\begin{proof}
				Taylor-series
			\end{proof}
		\end{corollary}
		Even tough the above results are nice, those theorems are non-uniform.
		The theorems only say that, when $f$ is polynomial time computable, then there exists 
		a polynomial time computable coefficient sequence and if there is a polynomial time computable sequence, then
		there is some algorithm that can compute the series in polynomial time.
		In no way, however, do they say, how to compute the sequence from a given representation of the function and vice versa.
		In fact the following theorems show, that this is not a problem of those theorems, but it's inherently impossible to do this.
		\begin{theorem}[M\"uller \cite{Mue}]
			Let $f$ given as in ... then the operator $f \to (a_n)_{n \in \NN}$ computing the Taylor series around $0$ (or any other point) is not computable.
		\end{theorem} 
		\begin{theorem}[M\"uller \cite{Mue}]
			Let $(a_n)_{n \in \NN}$ be the series expansion around $0$ for some $f \in C^\omega(D)$.\\
			The evaluation operator $((a_n)_{n \in \NN}, x) \to f(x)$ that, given a series and a point, computes the value of the corresponding function at that point, is not computable.
			\begin{proof}
				One does not know how many coefficients needed...
			\end{proof}
		\end{theorem} 

	\subsection{Representation of Analytic Functions}
	 As seen in the previous section, the information given by the series expansion is not enough to represent an analytic function.
	 However, by enriching the information by some finite discrete parameters, the translation between Taylor-series and function representation can be made uniform.
	 In particular the following two representation for analytic functions are useful
	 \begin{definition}\label{def:series_name_ball}
	 	A \textbf{series-name} $\rho_s$ of $f \in C^\omega(\overline{B_1(0)})$ is a triple $((a_n)_{n \in \NN}, k, A)$ where 
	 	\begin{enumerate}
	 		\item $(a_n)_{n \in \NN}$ is the series expansion of $f$ around $0$
	 		\item $\sqrt[k]{2} \leq R$ 
	 		\item $|a_j|r^j \leq A$ for all $j \in \NN$
	 	\end{enumerate}
	 	where $R = (\limsup |a_j|^{\frac{1}{j}})^{-1}$ denotes the radius of convergence of the series.
	 \end{definition}
	 Those two additional parameters can be used to make a tail estimate 
	 \begin{equation}\label{eqn:tail_estimate}
	  \left | \sum_{n \geq N} a_nx^n \right | \leq A \frac{(|z|/r)^N}{1-|z|/r}
	 \end{equation}
	 A name as in Definition \ref{def:series_name_ball} can be found by choosing any appropriate $k$ (note that the radius of convergence is always bigger than $1$) and choosing $A$ as an upper bound of $f$ extended to $\overline{B_{\sqrt[k]{2}}(0)}$. 
	 \begin{definition}
	 	A \textbf{function-name} $\rho_f$ of $f \in C^\omega(\overline{B_1(0)})$ is a triple $(f, l, B)$ such that
		B is an upper bound of $f$ on $\overline{B_{\sqrt[l]{2}}(0)}$.
	 \end{definition}
	 \begin{theorem}\cite{Kaw}\label{thm:representation_conversion}
	 	The mapping between $\rho_s$-name and $\rho_f$-names is computable in time polynomial in 
	 	$n+k+\log(A)$ and the inverse mapping is computable in time polynomial in $n+l+\log(B)$ 
	 \end{theorem}
	 \begin{proof}
	 	To compute a function name from a series name it only has to be shown that evaluation...

	 	To compute a series name from a function name, one has to compute the coefficients of the series expansion from a function name, i.e. from the function and the parameters $l$ and $B$.
	 	In \cite{Mue} M\"uller describes an algorithm for this task.
	 	To approximate the coefficient $a_k$ with precision at least $2^{-n}$ $f$ is approximated by the Lagrangian interpolation
	 	polynomial
	 	\begin{equation}\label{eqn:interpolation_polynomial}
	 		P_m(x)  :=  \sum_{i=0}^{2m} f(x_i) \cdot L_{m,i}(x) 
	 	\end{equation}
	 	where
	 	\begin{eqnarray*}
	 		x_i & = & (i-m) \cdot h , h \in \RR, h > 0 \\
	 		L_{m,i}(x) & = & \prod_{i \neq j} \frac{x-x_j}{x_i-x_j} 
	 	\end{eqnarray*}
	 	Differentiating Equation \ref{eqn:interpolation_polynomial} $k$ times yields
	 	\begin{equation}\label{eqn:interpolation_polynomial_diff}
	 		P_m^k(0)  :=  \sum_{i=0}^{2m} f(x_i) \cdot L_{m,i}(x) 
	 	\end{equation}
	 \end{proof}
	 Note that the above translation becomes fully polynomial time when in the representations $k$ resp. $l$ are required to be encoded in unary, while $A$ resp. $B$ are given in binary.
	 Thus, from now on the term polynomial time computable will be used when referring to running time bounds as in Theorem \ref{thm:representation_conversion}.
	 \begin{theorem}\label{thm:polytime_on_ball}
	 	The following is polynomial time computable when given $\rho_s$ or $\rho_f$ names for the input functions.
	 	\begin{enumerate}
	 		\item Evaluation $(f,z) \to f(z)$
	 		\item Addition $(f_1, f_2) \to f_1 + f_2$
	 		\item Multiplication $(f_1, f_2) \to f_1 \cdot f_2$
	 		\item $d$-fold Differentiation $(f,d) \to f^{(d)}$ where $d$ is given as unary
	 		\item $d$-fold Anti-differentiation $(f,d) \to \int \dots \int f$ where $d$ is given in unary
	 		\item Parametric maximization
	 	\end{enumerate}
	 	\begin{proof}(Sketch)
	 		Evaluation follows from \ref{thm:representation_conversion}. \\
	 		Except for parametric maximization the transformation of the Taylor series is obvious (e.g. for addition, just add the coefficients).
	 		It remains to show how to compute the new parameters $A'$ and $k'$. 
	 	\end{proof}
	 \end{theorem}
	Now consider functions analytic on some other domain, e.g. on the real line $[0,1]$.
	Such functions can be represented by a finite number of powerseries covering the domain. 
	That leads to the following representation
	\begin{definition}\label{def:series_name_rect}
		A \textbf{series-name} for a function $f \in C^\omega([0,1])$ is a 5-tuple $(M, (x_m), (a_{n, j}), k, A)$ where $M \in \NN$
		$1 \leq j \leq M$, $n \in \NN$, $x_m \in [0,1]$ and it holds
		\begin{enumerate}
			\item $[0,1] \subseteq \bigcup_{m=1}^M [x_m - \frac{i}{4k}, x_m + \frac{1}{4k}]$
			\item $(a_{n,i})_{n \in \NN}$ is the series expansion of $f$ around $x_0 + \frac{i}{4k}$
			\item $|a_{n,i}| \leq Ak^n$ for all $n \in N$, $1 \leq i \leq M$
		\end{enumerate}
	\end{definition}
	A function name as in Definition \ref{def:function_name_ball} can also be defined
	\begin{definition}
		Let $R_l := [-\frac{1}{l}, 1+frac{1}{l}] \times [-\frac{1}{l}, frac{1}{l}]$ the closed rectangle around $[-1,1]$ 
		with distance $\frac{1}{l}$ around the line $[0,1]$.
		A \textbf{function-name} for a function $f \in C^\omega([0,1])$ is a 3-tuple $(f|_{[-1,1]}, B, l)$ with $B, l \in \NN$ such that 
		\begin{enumerate}
			\item $f \in C^\omega(R_l)$
			\item $B$ is an upper bound for $f$ on $R_l$
		\end{enumerate}
		Again, $l$ is coded in unary and $B$ in binary.
	\end{definition}
	Again it can be shown that the two representations are polynomial time equivalent, and that the equivalent to
	Theorem \ref{thm:polytime_on_ball} holds.
	\subsection{Analytic Continuation}
		One problem with the representation in Definition \ref{def:series_name_rect} is that when thinking of this representation as an interface for a datatype, it is very cumbersome for the user to provide all the information needed. 
		One has to give enough Taylor series to cover the whole interval.
		As said before, there is no real gain of information by providing all those series.
		The analytic function is uniquely defined by the Taylor series around a single point.
		Thus, one would like to have something like the following representation
		\begin{definition}
			A \textbf{single series name} for $f \in C^\omega([0,1])$ is a quadruple $(x_0, (a_n)_{n \in \NN}, k, A)$ such that
			\begin{enumerate}
				\item $(a_n)_{n \in \NN}$ is the series expansion of $x$ around $x_0$
				\item $A$ and $k$ are such that $| a_m | \leq Ak^m$ 
			\end{enumerate}
		\end{definition}
		A series name can be computed from a single series name by computing the series expansion on another point on the domain and then iterating this process until the series cover $[0,1]$ (see Figure \ref{fig}).
		Since the parameters $A$ and $k$ are such that they hold on the entire rectangle, 
		the new series will be valid on a ball with the same radius as the original series, effectly extending the domain.
		The series can be computed either by applying the algorithm from the proof of Theorem \ref{thm:representation_conversion} or by directly computing the derivatives at an other point using \ref{thm:polytime_on_ball}.
		
	\subsection{Class Overview}
	\subsection{Implementation Details}
	\subsection{Evaluation}
	\subsection{Applications}