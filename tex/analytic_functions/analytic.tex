%!TEX root = ../../thesis.tex
\section{A Data Structure for Analytic Functions}
	Acknowledgment: Florian, Akitoshi
	\subsection{Introduction}
		As seen in Chapter \ref{section:real_complexity} many important operators 
		on real functions have been shown to be computationally hard.
		On the other hand, numerical scientists can quickly compute maxima or anti-derivatives of functions.
		To analyze this mismatch one should ask the follwing questions
		\begin{enumerate}
			\item What subset of real functions is usually considered in numerical analysis?
			\item Are there implicit assumptions made about the functions, that allow faster computations? 
		\end{enumerate}
		One class of functions that has been considered in computable analysis is the analytic functions.
		Although not at all sufficent as the class of functions considered interesting for numerical analysis, 
		many computations become tractable when one restricts the attention to analytic functions.
		\begin{definition}
			For $z \in \CC$ and $r \in \RR, r > 0$, let $B(z,r) := \{ x \in \CC \,\, | x - z | < r\}$. \\
			A function $f: D \to \CC$ with $D \subseteq CC$ is called \textbf{analytic} if for every $x_0 \in D$ 
			the function given by the Taylor-series around $x_0$ converges to $f(x)$ for every $x$ in a neighborhood of 
			$x_0$. 
			That is, for every $x_0 \in \D$ there is a $\varepsilon > 0$, such that for all $x \in B(x_0, \varepsilon)$ and for 
			$$ T(x) := \sum_{n=0}^{\infty} \frac{f^{(n)}(x_0)}{n!}(x-x_0)^n  \text{, } T(x) \rightarrow f(x)$$.
			The set of functions analytic on $D$ is denoted as $C^\infty(D)$. \\
			The coefficients of the Taylor expansion is denoted by the series $(a_n)_{n \in \NN}$, i.e. 
			$$ a_n = \frac{f^{(n)}(x_0)}{n!}. $$    
			For the sake of simplicity, it will w.l.o.g. be assumed that $x_0 = 0$  and $D = [0,1]$ if not stated otherwise.
		\end{definition}
		Note: The above definition can be easily generalized to higher dimension. \\
		Some properties of analytic functions...
		The complexity of analytic functions is much better then for the general case.
		\begin{theorem}[Pour-El, Richards, Ko, Friedman, Müller] 
			The following are equivalent
			\begin{enumerate}
				\item $f$ is computable 
				\item The series $(a_n)_{n \in \NN}$ is computable. 
 			\end{enumerate}
 			The equivalence also holds, when computable is replaced by polynomial time computable.
		\end{theorem}
		From that it follows
		\begin{corollary}
			If $f$ is polynomial time computable the following functions are
			\begin{enumerate}
				\item $I: [0,1] \to \RR$, $I(x) = \int_0^x f(t) dt$
				\item $D: [0,1] \to \RR$, $D(x) = f'(x)$ 
			\end{enumerate}
			\begin{proof}
				Taylor-series
			\end{proof}
		\end{corollary}
		Even tough the above results are nice, those theorems are non-uniform.
		The theorems only say that, when $f$ is polynomial time computable, then there exists 
		a polynomial time computable coefficient sequence and if there is a polynomial time computable sequence, then
		there is some algorithm that can compute the series in polynomial time.
		In no way, however, do they say, how to compute the sequence from a given representation of the function and vice versa.
		In fact the following theorems show, that this is not a problem of those theorems, but it's inherently impossible to do this.
		\begin{theorem}[M\"uller \cite{Mue}]
			Let $f$ given as in ... then the operator $f \to (a_n)_{n \in \NN}$ computing the Taylor series around $0$ (or any other point) is not computable.
		\end{theorem} 
		\begin{theorem}[M\"uller \cite{Mue}]
			Let $(a_n)_{n \in \NN}$ be the series expansion around $0$ for some $f \in C^\omega(D)$.\\
			The evaluation operator $((a_n)_{n \in \NN}, x) \to f(x)$ that, given a series and a point, computes the value of the corresponding function at that point, is not computable.
			\begin{proof}
				One does not know how many coefficients needed...
			\end{proof}
		\end{theorem} 

	\subsection{Representation of Analytic Functions}
	 As seen in the previous section, the information given by the series expansion is not enough to represent an analytic function.
	 However, by enriching the information by some finite discrete parameters, the translation between Taylor-series and function representation can be made uniform.
	 In particular the following two representation for analytic functions are useful
	 \begin{definition}\label{def:series_name_ball}
	 	A \textbf{series-name} $\rho_s$ of $f \in C^\omega(\overline{B_1(0)})$ is a triple $((a_n)_{n \in \NN}, k, A)$ where 
	 	\begin{enumerate}
	 		\item $(a_n)_{n \in \NN}$ is the series expansion of $f$ around $0$
	 		\item $\sqrt[k]{2} \leq R$ 
	 		\item $|a_j|r^j \leq A$ for all $j \in \NN$
	 	\end{enumerate}
	 	where $R = (\limsup |a_j|^{\frac{1}{j}})^{-1}$ denotes the radius of convergence of the series.
	 \end{definition}
	 Those two additional parameters can be used to make a tail estimate 
	 \begin{equation}\label{eqn:tail_estimate}
	  \left | \sum_{n \geq N} a_nx^n \right | \leq A \frac{(|z|/r)^N}{1-|z|/r}
	 \end{equation}
	 A name as in Definition \ref{def:series_name_ball} can be found by choosing any appropriate $k$ (note that the radius of convergence is always bigger than $1$) and choosing $A$ as an upper bound of $f$ extended to $\overline{B_{\sqrt[k]{2}}(0)}$. 
	 \begin{definition}
	 	A \textbf{function-name} $\rho_f$ of $f \in C^\omega(\overline{B_1(0)})$ is a triple $(f, l, B)$ such that
		B is an upper bound of $f$ on $\overline{B_{\sqrt[l]{2}}(0)}$.
	 \end{definition}
	 \begin{theorem}\cite{Kaw}\label{thm:representation_conversion}
	 	The mapping between $\rho_s$-name and $\rho_f$-names is computable in time polynomial in 
	 	$n+k+\log(A)$ and the inverse mapping is computable in time polynomial in $n+l+\log(B)$ 
	 \end{theorem}
	 Note that the above translation becomes fully polynomial time when in the representations $k$ resp. $l$ are required to be encoded in unary, while $A$ resp. $B$ are given in binary.
	 Thus, from now on the term polynomial time computable will be used when referring to running time bounds as in Theorem \ref{thm:representation_conversion}.
	 \begin{theorem}
	 	The following is polynomial time computable when given $\rho_s$ or $\rho_f$ names for the input functions.
	 	\begin{enumerate}
	 		\item Evaluation $(f,z) \to f(z)$
	 		\item Addition $(f_1, f_2) \to f_1 + f_2$
	 		\item Multiplication $(f_1, f_2) \to f_1 \cdot f_2$
	 		\item $d$-fold Differentiation $(f,d) \to f^{(d)}$ where $d$ is given as unary
	 		\item $d$-fold Anti-differentiation $(f,d) \to \int \dots \int f$ where $d$ is given in unary
	 		\item Parametric maximization
	 	\end{enumerate}
	 \end{theorem}
	\subsection{Class Overview}
	\subsection{Implementation Details}
	\subsection{Evaluation}
	\subsection{Applications}