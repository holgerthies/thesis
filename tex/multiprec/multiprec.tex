%!TEX root = ../../thesis.tex
\section{Arbitrary-Precision Arithmetic}
	Often the precision of hardware supported floating point datatypes is not sufficent (example?).
	
	When precise results are more important than speed of computation, Arbitrary-precision arithmetic can be used.

	Arbitrary precision means that the user is able to choose the precision with which 
	computations are performed (usually only limited by the available main memory).

	Libraries for arbitrary-precision arithmetic are available for many modern programming languages, 
	e.g. the GNU MPFR Library for \code{C}/\cc, The Boost Multiprecision Library for \cc or \code{mpmath} for \code{Python}. 

