%!TEX root = ../../thesis.tex
\section{Arbitrary-Precision Arithmetic}
	The accuracy that can be obtained by using hardware floating point types
  sometimes does not suffice.
	
	When precise results are more important than speed of computation, Arbitrary-precision arithmetic can be used.

	Arbitrary precision means that the user is able to choose the precision with which 
	computations are performed (usually only limited by the available main memory).
  Usually an arbitarty-precision number is implemented as a list or array of
  built-in data-structures together with algorithms that perform arithmetic operations on
  this representation.

	Many libraries for arbitrary-precision arithmetic exist for nearly all modern
  programming languages.
  
	Examples are the GNU MPFR Library for \code{C}/\cc, The Boost Multiprecision Library for \cc or \code{mpmath} for \code{Python}. 
  
  The floating point types defined by such libraries can have a user-defined
  length. 
  Thus, the precision can be arbitrarily good, but will still be fixed during
  the program flow.
  The user has the responsibility to choose a precision high enough to
  guarantee, that the end result's precision will suffice.

