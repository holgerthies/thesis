%!TEX root = ../../thesis.tex
\section{Arbitrary-Precision Arithmetic}
	The accuracy that can be obtained by using hardware floating point types
  sometimes does not suffice.
	
	When precise results are more important than speed of computation, Arbitrary-precision arithmetic can be used.

	Arbitrary precision means that the user is able to choose the precision with which 
	computations are performed (usually only limited by the available main memory).

  Usually an arbitrary-precision number is implemented as a list or array of
  built-in data-structures together with algorithms that perform arithmetic operations on
  this representation.

	Many libraries for arbitrary-precision arithmetic exist for nearly all modern
  programming languages.
  
	Examples are the GNU MPFR Library for \code{C}/\cc \cite{mpfr}, The Boost Multiprecision
Library for \cc \cite{boostmultiprecision} or \code{mpmath} for \code{Python}
\cite{mpmath}. 
  
  The floating point types defined by such libraries can have a user-defined
  length. 
  Thus, the precision can be arbitrarily good, but will still be fixed during
  the program flow.
  Rounding errors still occur and in most implementations it is not possible to
  get bounds on the error from the computation result.

  The user usually has to perform additional analysis on the problem, to choose
  a computation precision that yields exact enough results.
