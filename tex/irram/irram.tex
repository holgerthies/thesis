%!TEX root = ../../thesis.tex
\section{iRRAM}
	\irram is a \cc framework for exact real arithmetic developed by Norbert M\"uller.

  The case studies in the next chapters were all implemented using the \irram
  framework. 
  This chapter gives therefore a short introduction to this framework.

	\irram extends \cc by a data-type \real for error free computations with real numbers.
  For the user, an object of type \real behaves like a real number that can be
  manipulated without any rounding errors.
  The framework takes care of all details necessary to finitely represent real
  numbers internally, and in most cases this internal representation will be invisible for
  the user.
	\subsection{Real Number representation}
		{\irram}'s approach is similar to the bottom-up DAG approach described in
    Section \ref{sec:exact real arithmetic}.

		There is one important difference tough: The DAGs are  stored only
    implicitly.
		Instead, there is only enough information saved, so that the whole
    computation can be repeated from the beginning.
    If the precision at some point of the program does not suffice, the whole
    program is restarted with higher precision. 
    This drastically reduces the memory needed for the computation.

    For making error analysis, \irram uses a simplified form of interval
    arithmetic.

		At every time during the program execution, the internal representation of
    a real number is always a \textbf{single} interval. 
    More precisely, a real number $x \in \RR$ is represented by a pair $(d,e)$ such that $x \in [d-e, d+e]$.

 		$d$ is saved as a multiple precision number and $e$ consists of two \code{long}s $p,z$ such that $e = p \cdot 2^p$.

    The internal representation of a \real can change during the run of the
    program.
 		
 	  Whenever the precision of some \real is not sufficient at some point, the whole computation is restarted.
    This is, when the representations of the reals change to admit higher precision.

 		This process of restarting the whole computation with higher precision is called
    \textbf{iteration} in {\irram}'s terminology.
    Since a reiteration is only done when higher precision is needed, there
    are only few operations, such as comparisons or output, that can trigger such a reiteration.

    An \irram program will usually do several iterations of the computation,
    until the desired output precision is reached.
    
	\subsection{Overview}
	  The following gives an overview over the most important classes and
    functions provided by the \irram framework. For a more detailed overview
    see \cite{irram}.
		
    \irram provides some classes for discrete valued structures, in particular
    the class \code{INTEGER} for infinite sized integers, the class
    \code{RATIONAL} for rational numbers and the class \code{DYADIC} for dyadic
    rational numbers. 

    The most important class is \real, the class for error-free computations
    with real numbers. 

		An object of type \real can be constructed from \code{int}, \code{char*}, \code{double}, \code{DYADIC} 
		or another \real. 

		Arithmetic operations can be applied on {\real}s by using the overloaded operators \code{+}, \code{-}, \code{*},
		\code{/}.

		Many functions on {\real}s are already included in the \irram package, e.g. 
		\code{abs(x)}, \code{power(x, n)}, \code{exp(x)}, \code{log(x)},
		trigonometric functions and their inverse and many more.

		The operators \code{<}, \code{<=}, \code{==}, \code{>=}, \code{>} are also
    overloaded.
    Note, however, that comparison operators have to be used carefully, as will be explained in the
    next section.
    
   \irram further provides classes for complex numbers and real matrices.

		An \irram program is written in usual \cc extended by the data-types provided by a framework.
		However, a few things have to be kept in mind when writing a program
		\begin{enumerate}
      \item \textbf{compute function}
        The \irram framework overwrites the \cc \code{main} function, so
        it can not be written by the user.
        Instead it is replaced by a function \code{void iRRAM :: compute() }.
        The main part of the program should be written inside this function. 
      \item \textbf{Input and Output}
        The input and output streams \code{cin} and \code{cout} are overloaded
        with {\irram}'s own versions, that can read and write real numbers. 
        The precision of the output can be adjusted with the parameterized
        manipulator \code{setrwidth} that sets the number of digits that are
        written. 
        Alternatively the function \code{rwrite(REAL\& r, int prec)} can be used
        to write a real number with a given number of decimals.
      \item \textbf{Global variables} Since \irram only reiterates the code
        executed inside the compute function, global variables will not be part of the
        reiteration. Therefore, \real variables should never be defined
        globally.
		\end{enumerate}
	\subsection{Multivalued Functions}
		TTE can be used as a theoretical model to reason about \irram programs.

		Thus, the limitations seen in Chapter \ref{sec:real computability} also hold for \irram programs.  
		In particular, every function computable with \irram necessarily has to be continuous.
		
		That also implies, that comparisons and tests for equality of {\real}s are not computable.

		Indeed, for the simplified representation a comparison of two numbers can be decided
    if and only if the intervals of the two representations do not overlap.

	  As long as the two intervals overlap, \irram has to assume that the
    intervals are to large to make a decision yet, and has to reiterate to
    get smaller intervals.
    But if the numbers are equal, the intervals will always overlap, 
		no matter how small they are, leading to an infinite loop of iterations.
    \begin{table}
       \centering
       \begin{tabular}{|c||c|c|c|}
        \hline
        $ a || b$ & $0$ & $1$ & $\bot$ \\ \hhline{|=||=|=|=|} 
        $ 0 $     & $0$ & $1$ & $\bot$ \\ \hline
        $ 1$      & $1$ & $1$ & $1$ \\ \hline
        $\bot$ & $\bot$  & $1$ & $\bot$ \\ \hline
       \end{tabular}
       \quad
       \begin{tabular}{|c||c|c|c|}
        \hline
        $ a \&\& b$ & $0$ & $1$ & $\bot$ \\ \hhline{|=||=|=|=|} 
        $ 0 $     & $0$ & $0$ & $0$ \\ \hline
        $ 1$      & $0$ & $1$ & $\bot$ \\ \hline
        $\bot$ & $0$  & $\bot$ & $\bot$ \\ \hline
       \end{tabular}
       \quad
       \begin{tabular}{|c||c|}
        \hline
        $ !a $ &   \\ \hhline{|=||=|} 
        $ 0 $ &  1 \\ \hline
        $ 1$ &  0 \\ \hline
        $\bot$ & $\bot$  \\ \hline
       \end{tabular}
    \caption{The semantics of the \code{LAZY\_BOOLEAN} data-type}\label{table:lazy
  boolean}
    \end{table}
    The reiteration is realized by the \code{LAZY\_BOOLEAN} data-type.
    \code{LAZY\_BOOLEAN} extends the \code{boolean} type to  ternary logic
		with values $T, F,\bot$. The semantics of \code{LAZY\_BOOLEAN} can be found
    in Table \ref{table:lazy boolean}.
    
    The informal meaning of $\bot$ is, that the precision is not yet sufficient
    to make a decision.
		When casting to \code{bool} $\bot$ leads to a reiteration. 

    Not being able to make comparisons might first seem like a huge
    disadvantage, but note that in numerical practice it is also strongly
    advised to never test floating point numbers on equality.  The \code{gnu
    gcc} compiler even has a warning option on that \cite{GCCMan}:
		\begin{quotation}
				\noindent
		\begin{verbatim}
			-Wfloat-equal
			  Warn if floating-point values are used in equality comparisons.

			  The idea behind this is that sometimes it is convenient (for the programmer) to consider 
			  floating-point values as approximations to infinitely precise real numbers. [...]
			  Instead of testing for equality, you should check to see whether the two values 
			  have ranges that overlap; and this is done with the relational operators, so equality 
			  comparisons are probably mistaken. 
		    \end{verbatim}
		\end{quotation} 

		In contrast to that \irram has a semantically correct way to handle comparisons and other
		cases of non-continuity: multi-valued functions.

		In \irram a multi-valued function can have multiple valid outputs for the same input and 
		return one of the valid outputs.
    For the user the choice of output can seem indeterministic. 
    Of course, the output will be deterministic in the underlying
    finite representation of the number, but the representation might differ even when
    the real number represented is the same.
		
		\irram already has some multi-valued functions included, e.g. the function
		\begin{equation*}
			bound(x,k) = 
			\begin{cases}
				\code{true} & \mbox{if } | x | \leq 2^{k-2} \\
				\code{true} \mbox{ or } \code{false} & \mbox{if } 2^k \geq |x| > 2^{k-2} \\
				 \code{false}  & \mbox{if } |x| > 2^k.
			\end{cases}
		\end{equation*}
		can be used for multi-valued comparisons.

		Another example is the function \code{approx(const REAL\& x, const long p)} that returns a \code{DYADIC} 
		approximation with error less than $2^{-k}$ to $x$.	

		One way to construct multi-valued functions without accessing the internal representation of the {\real}s 
		is to use \code{LAZY\_BOOLEAN} together with the \code{choose}-function.
		\code{choose} is a multi-valued function that takes up to 6 lazy booleans and returns the index of one 
		that is evaluated to $T$ or $0$ if all evaluate to $F$ (leading to reiteration as long as neither of the two cases hold).

		The powerful feature to have multi-valued functions introduces, however, a
    new problem.

		A multi-valued function can have different results for different representations of the same real number.
    Since the internal representation of a real number changes when the
    computation is reiterated, a reiteration could lead to a completely
    different result of the multi-valued function.
		Thus, the program flow could change from one iteration to another, leading to unexpected results.

		To prevent that from happening, the result of a multi-valued function is always saved in the so called \textbf{multi-value cache}
		and read from there when reiterating. 
		It is therefore guaranteed that the program flow stays the same in all iterations.
		
    Saving everything in the cache will, however, increase the needed memory.
    Multi-valued functions should only be used rarely and one has to be
    especially careful when they are used inside of loops.
	\subsection{Limits}
		\irram also has the feature to generate new user-defined {\real}s by using a special \code{limit} operator.
		There are three types of limits
		\begin{enumerate}
			\item \textbf{Simple limits}: \\
			This limit has the form
			\begin{verbatim}
				REAL limit(REAL a(long, const REAL&), const REAL& x).
			\end{verbatim}

			The limit operator takes a sequence of real functions converging to a single-valued function $f: \RR \to \RR$
			rapidly, i.e. such that 
			$$|a_p(x) - f(x)| \leq 2^p \text{ for all } x \in dom(f)$$
			and returns the value of $f(x)$ as a \real.

			There exist similar functions to compute limits of sequences of functions with $0$ or $2$ input variables.
			\item \textbf{Lipschitz limits} \\
				The limit computation can be improved if a Lipschitz bound on the limit function $f$ is known, i.e. one has an $l$
				such that
				\begin{equation*}
					| f(x) - f(y) | \leq 2^l | x - y | \text{ for } x,y in dom(f).
				\end{equation*}
				Then for $x \in [d-e, d+e]$  
				$$ | a_p(d) - f(x) | \leq | a_p(d) - f(d) | + | f(d) - f(x) |  \leq 2^p + 2^l \cdot e $$
				Thus for large enough $p$ (i.e. such that $2^p > 2^l \cdot e$)
				$$ | a_p(d) - f(x) | \leq  2^{p+1} $$
				To use this in \irram there is a function
				\begin{verbatim}
					REAL limit_lip(REAL a(long, const REAL&), long lip, const REAL& x).
				\end{verbatim}
			\item \textbf{Multi-valued limits}
      The multi-valued limit-operator has the form 
      \begin{verbatim}
        REAL limit(REAL f(int prec, int* choice, const REAL&), const REAL&)
      \end{verbatim}
      The parameter choice selects a subset of values that are possible for the
      limit of $f$. $0$ means that all values are allowed, and changing the
      value of choice corresponds to a reduction of the possible values.
      The choice parameter will be saved in the multi-value cache.
		\end{enumerate}
	\subsection{Access to the underlying representation}
		Usually the user is expected to see the data-type \real as a kind of
    black-box that makes error-free computations with real numbers possible.
    The framework should take care of all the underlying representations and
    the finite representation should be invisible to the user.

    In rare cases it may however be helpful, to access and change the
    underlying representation of {\real}s.
    This is also possible in \irram.
     
		The error of a \real has the following type
		\begin{verbatim}
			struct sizetype { SIZETYPEMANTISSA mantissa; SIZETYPEEXPONENT exponent; }
		\end{verbatim}
		where \code{SIZETYPEMANTISSA} is \code{unsigned int} and \code{SIZETYPEEXPONENT} is int.
		
		An object of type \real has the two member functions
		\begin{verbatim}
			void seterror (sizetype error);
			void geterror (sizetype& error) const;
		\end{verbatim}
    They can be used to get and set the error.
    
    The information about the current iteration can also be accessed.
    The most notable variable is \code{ACTUAL\_STACK.ACTUAL\_PREC} that
    contains the precision bound for the current iteration.
