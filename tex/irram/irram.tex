%!TEX root = ../../thesis.tex
\section{iRRAM}
	Since the case studies in this thesis were all implemented  in \irram, the \irram framework 
	is discussed in detail.

	\irram is a \cc framework for exact real arithmetic devoloped by Norbert M\"uller.
	\irram extends \cc by a datatpe \real for error free computations with real numbers.
	\subsection{Real Number representation}
		\irram's approach is similiar to the bottom-up DAG approach described in Section \ref{sec:exact real arithmetic}
		with the difference that the DAGs are not really constructed.

		Instead there is only enough information saved to repeat the whole computation when the precision is not sufficent, and thus reducing the needed memory to a minimum.

		\irram uses simplified interval arithmetic for real number computations.
		During the run of the program, a real number always represented by a \textbf{single} interval, more precisely by a pair $(d,e)$ such that $x \in [d-e, d+e]$.
 		$d$ is saved as a multiple precision number and $e$ consists of two \code{long}s $p,z$ such that $e = p \cdot 2^p$.
 		Whenever at some point in the program the precision is not sufficent, the computation is restarted with higher precision, 
 		changing the interval representations of the real numbers.
 		Thus, computations in \irram are done iteratively, where each iteration increases the precision of the output and the process is terminated when the desired output precision is reached. 
	\subsection{Overview of Classes}
		
	\subsection{Multivalued Functions}
	\subsection{Limits}

