%!TEX root = ../../thesis.tex
\section{iRRAM}
	Since the case studies in this thesis were all implemented  in \irram, the \irram framework 
	is discussed in detail.

	\irram is a \cc framework for exact real arithmetic devoloped by Norbert M\"uller.
	\irram extends \cc by a datatype \real for error free computations with real numbers.
	\subsection{Real Number representation}
		\irram's approach is similiar to the bottom-up DAG approach described in Section \ref{sec:exact real arithmetic}
		with the difference that the DAGs are not really constructed.

		Instead there is only enough information saved to repeat the whole computation when the precision is not sufficent, and thus reducing the needed memory to a minimum.

		\irram uses simplified interval arithmetic for real number computations.
		During the run of the program, a real number always represented by a \textbf{single} interval, more precisely by a pair $(d,e)$ such that $x \in [d-e, d+e]$.
 		$d$ is saved as a multiple precision number and $e$ consists of two \code{long}s $p,z$ such that $e = p \cdot 2^p$.
 		Whenever at some point in the program the precision is not sufficent, the computation is restarted with higher precision, 
 		changing the interval representations of the real numbers.
 		Thus, computations in \irram are done iteratively, where each iteration increases the precision of the output and the process is terminated when the desired output precision is reached. 

 		The theoretical model to reason about \irram programs is TTE.
	\subsection{Overview}
		The following section introduces the most important classes and functions of \irram. 
		For a complete see
		
		An \irram program is a usual \cc program.
		\irram provides the class \real for error free computations with real numbers.
		An object of type \real can be constructed from \code{int}, \code{char*}, \code{double}, \code{DYADIC} 
		or another \real. 

		Arithmetic operations can be applied on \real s by using the overloaded operators \code{+}, \code{-}, \code{*},
		\code{/}.

		Many functions on \real s are already included in the \irram package, e.g. 
		\code{abs(x)}, \code{power(x, n)}, \code{exp(x)}, \code{log(x)},
		trigonometric functions and their inverse and many more.

		The operators \code{<}. \code{<=}, \code{==}. \code{>=}, \code{>} are also overloaded, but have to be used carefully.


	\subsection{Multivalued Functions}
		As said before, the theoretical model to reason about \irram programs is TTE.
		Thus, the limitations seen in Chapter \ref{sec:real computability} also hold for \irram programs.  
		In particular, every function computable with \irram is necesserily continuous.
		
		That also implies, that comparisons and tests for equality of {\real}s are not computable when the input numbers are equal.

		Indeed, two numbers can only be compared if the intervals they are represented by do not overlap. 
		Thus, if the intervals overlap, \irram has to reiterate. But if the numbers are equal, the intervals will always overlap, 
		no matter how small they are, leading to an infinite loop of iterations.
		
		In practice, this is achieved by letting those functions have type \code{LAZY\_BOOLEAN}.
 		\code{LAZY\_BOOLEAN} realizes ternary logic
		$T, F,\bot$, where $\bot$ means that the current information is not sufficent to make a decision.
		When casting to \code{bool} $\bot$ leads to reiteration. The semantics of \code{LAZY\_BOOLEAN} can be found in Table.

		Not being able to make comparisons might first seem like a huge disadvantage, but note that in numerical practice it is also strongly advised to never test floating point numbers on equality.
		The \code{gnu gcc} compiler even has a warning option on that:
		\begin{quotation}
				\noindent
		\begin{verbatim}
			-Wfloat-equal
			  Warn if floating-point values are used in equality comparisons.

			  The idea behind this is that sometimes it is convenient (for the programmer) to consider 
			  floating-point values as approximations to infinitely precise real numbers. [...]
			  Instead of testing for equality, you should check to see whether the two values 
			  have ranges that overlap; and this is done with the relational operators, so equality 
			  comparisons are probably mistaken. 
		    \end{verbatim}
		\end{quotation} 

		In contrast to that \irram has however, a sementically correct way to handle comparisons and other
		cases of non-continuity: multi-valued functions.

		In \irram a multi-valued function can have multiple valid outputs for the same input and 
		return seemingly indeterministic one of the valid outputs (of course, everything is deterministic in the 
		interval representation of the input numbers, but that is usually not observed by the user).
		
		\irram already has some multi valued functions included, e.g. the function
		\begin{equation*}
			bound(x,k) = 
			\begin{cases}
				\code{true} & \mbox{if } | x | \leq 2^{k-2} \\
				\code{true} \mbox{ or } \code{false} & \mbox{if } 2^k \geq |x| > 2^{k-2} \\
				 \code{false}  & \mbox{if } |x| > 2^k.
			\end{cases}
		\end{equation*}
		can be used for multi-valued comparisons.

		Another example is the function \code{approx(const REAL\& x, const long p)} that returns a \code{DYADIC} 
		approximation with error less than $2^{-k}$ to $x$.	

		One way to construct multi-valued functions without accessing the internal representation of the {\real}s 
		is to use \code{LAZY\_BOOLEAN} together with the \code{choose}-function.
		\code{choose} is a multi-valued function that takes up to 6 lazy booleans and returns the index of one 
		that is evaluated to $T$ or $0$ if all evaluate to $F$ (leading to reiteration as long as neither of the two cases hold).

		The possibility to have multi-valued functions introduces a new problem: 
		A multi-valued function can have different results for different representations of the same real number,
		thus the outcome could also change in different iterations of the \irram. 
		Thus, the program flow could be completely change from one iteration to another, leading to unexpected results.
		To prevent that from happening, the result of a multi-valued function is saved in the so called \textbf{multi-value cache}
		and read from there when reiterating. 
		Thus, it is guaranteed that the program flow stays the same in each iteration.
		This makes, however, the use of multi-valued functions memory-intensive and they should therefore be used with care,
		especially inside loops.
	\subsection{Limits}
		\irram also has the feature to generate new user-defined {\real}s by using a special \code{limit} operator.
		There are three types of limits
		\begin{enumerate}
			\item \textbf{Simple limits}: \\
			This limit has the form
			\begin{verbatim}
				REAL limit(REAL a(long, const REAL&), const REAL& x).
			\end{verbatim}

			The limit operator takes a sequence of real functions that converges to a single-valued function $f: \RR \to \RR$
			rapidly, i.e. such that 
			$$|a_p(x) - f(x)| \leq 2^p \text{ for all } x \in dom(f)$$
			and returns the value of $f(x)$ as a \real.

			There exist similiar functions to compute limits of sequences of functions with $0$ or $2$ input variables.
			\item \textbf{Lipschitz limits} \\
				The limit computation can be improved if a Lipschitz bound on the limit function $f$ is known, i.e. one has an $l$
				such that
				\begin{equation*}
					| f(x) - f(y) | \leq 2^l | x - y | \text{ for } x,y in dom(f).
				\end{equation*}
				Then for $x \in [d-e, d+e]$  
				$$ | a_p(d) - f(x) | \leq | a_p(d) - f(d) | + | f(d) - f(x) |  \leq 2^p + 2^l \cdot e $$
				Thus for large enough $p$ (i.e. such that $2^p > 2^l \cdot e$)
				$$ | a_p(d) - f(x) | \leq  2^{p+1} $$
				To use this in \irram there is a function
				\begin{verbatim}
					REAL limit_lip(REAL a(long, const REAL&), long lip, const REAL& x).
				\end{verbatim}
			\item \textbf{Multi-valued limits}
		\end{enumerate}
	\subsection{Access to the underlying representation}
		
