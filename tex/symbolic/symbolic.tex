%!TEX root = ../../thesis.tex
\section{Symbolic Computation}
Symbolic computation means that mathematical expressions (i.e. formulas consising of numbers, functions, variables and mathematical constants) are manipulated algebraically.
Software that performs symbolic computations is called a \textbf{Computer Algebra System} (CAS).

Some example of Computer Algebra Systems include Maple, Mathematica or Sage.

In a CAS, at each step of the computation the numbers are represented exactly.

However, symbolic computation is very limited, because many problems are either 
very difficult to be solved symbolically or do not even have a symbolic solution at all. 

Also, even if the problem can be solved an exact symbolic solution might be very long and complex, 
thus making it less useful on its own. 

Therefore, symbolic methods are often combined with numerical methods, 
to get an approximate result in the end.
