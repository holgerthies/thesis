A large amount of problems that are solved with the help of computers are
problems dealing with real numbers.
Examples include computing zeros, maxima or minima of real functions, solving
differential equations or determining eigenvalues.

The most common way to represent real numbers on a computer is the use of floating point
arithmetic.
The floating point representation of a real number is a finite word, consisting
of a mantissa and an exponent of fixed length.
Due to the finiteness of the representation, it is unavoidable that errors
occur.
While in many cases the error stays small, there are also several examples
where computations with floating point numbers yield results that are
completely wrong and far off from the correct value.

In any case, the error is not visible from the representation, thus without
additional analysis it is impossible to know if the error is too big for a
particular application or not.
Floating point arithmetic is therefore not a reliable way to perform real number
computations.

There are many cases where reliable results are necessary.
Writing reliable algorithms on numerical problems requires, however, a sound
theoretical foundation.

While every natural number is obviously computable, the same can not hold for
real numbers for countability reasons.

Computable real numbers were already introduced by Alan Turing in his famous
paper "On computable numbers, with an application to the Entscheidungsproblem"
\cite{turing36}.
However, while for discrete-valued problems there is a widely accepted theory for
computability and complexity, there are several non-equivalent theories for
computations on real numbers and none of them is universally accepted among
researchers.

One such theory is the Type-2 Theory of Effectivity (TTE) \cite{Wei}.
In TTE reals are represented as infinite strings over some finite alphabet
$\Sigma$.
Informally one can say, a real number is computable in TTE if it is possible to approximate it with any
desired precision.

TTE is supposed to be a realistic model in the sense that exactly those
real numbers and functions are computable in TTE that can be computed with a
computer.

Similar to the discrete-valued case a vast theory on computability and
complexity has been built upon TTE.

Since TTE strives to be a realistic model, an important task is to compare
theoretical claims with practical implementations.
Using computers to perform exact computations on real numbers is called exact
real arithmetic.
There are already quite a few implementations for exact real arithmetic.
One that has been proven to be particularly fast and therefore well-suited for
practical implementations, is Norbert M\"uller's \cc framework \irram
\cite{irram}.

\irram extends \cc by classes and functions for error-free computations with
real numbers.
The details of the internal representation are taken care of by the framework. 
The user can write ordinary \cc extended by a data-type that behaves like
real numbers, without thinking about rounding errors.

This thesis presents two case studies in exact real arithmetic, both using the
\irram framework to implement the algorithms.

The first case-study shows how exact real arithmetic can be used to get simpler
algorithms for classical problems in numerical analysis.
The particular problem considered is the shadowing distance for chaotic
dynamic systems, i.e. how close a numerically computed orbit of a map $f: \RR
\to \RR$ is followed by an exact orbit.
This problem was originally considered by Hammel, Yorke and Grebogi in 1987
\cite{Hammel1987}. 
Since they could not compute real numbers exactly, they used a form of interval
arithmetic to bound the exact orbit.
However, \irram can be used to compute the exact orbit which thus yields a
simplified version of their algorithm.

The second case-study deals with analytic functions. 
Analytic functions have been thoroughly studied in real complexity theory as a
subset of real functions where many in general computationally hard problems become feasible. 
To show how well those complexity claims compare with practical
implementations, data-types to represent analytic functions have been written.
The implementation is meant as an extension to the \irram framework, that
provides user-friendly classes for computations with analytic functions.
Empirical evaluation was done on the running time of those classes and compared
with the expected running times from the theoretical examination.

