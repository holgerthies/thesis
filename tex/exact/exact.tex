%!TEX root = ../../thesis.tex
\section{Exact Real Arithmetic}
	Using interval arithmetic or arbitrary precision arithmetic, 
	the precision is set in the beginning and rounding and truncation errors still accumulate.
	In the end a result is obtained that may not be as precise as needed.

	In contrast, the goal of exact real arithmetic is to perform compuations on real numbers without 
	accumulating errors, and that can be given up to any desired precision on demand.
	That is, the user does not specify the precision the numbers are represented with in the beginning,
	but gives a desired precision \textbf{for the answer} of the computation.

	There are several approaches to exact real arithmetic.
	Some of them are presented below. 

	\subsection{Signed-digit streams}
	\subsection{Continued fractions}
	\subsection{Linear fractional transformations}
	\subsection{DAGs}
		An arithmetic expression over real numbers can be expressed by a directed acyclic graph (DAG)
		where the leaf nodes are real numbers and the inner nodes are operations on real numbers.
		
		Information in nodes??

		A DAG represents a real number and evaluating a DAG means to get an approximation to that real number
		with a desired precision.

		There are essentially two ways to evaluate a DAG:
		\begin{enumerate}
			\item \textbf{Top=down evaluation} means, that the desired precision is computed from the top node down to the leaves.
			That is, a node contains the information, how precise he needs the input given by its children to deliver the output with the necessary precision and requests its children to provide it.  
			\item \textbf{Bottom-up evaluation} means, that the computation is started with a fixed precision in the leaf nodes 
			and it is kept track of the errors (e.g. by using interval arithmetic). 
			If in the end, the error bound is too large, the computation is restarted with a higher precision. 
		\end{enumerate}
		The top-down approach suffers from the problem, that the precision can grow unnecessarily large.   
		Example.
	\subsection{Comparison}



