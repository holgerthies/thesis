\documentclass[tudarticle,type=msc,colorback,accentcolor=tud9c]{tudthesis}

  \usepackage[utf8]{inputenc}
  \usepackage[thmmarks]{ntheorem}
  \usepackage{xspace,cleveref,xcolor}
  \usepackage{amsmath}
  \usepackage{graphicx,url}
  \usepackage[font=small]{caption}
 \usepackage{subcaption}
 \usepackage[parfill]{parskip}
 \usepackage{newclude}
\usepackage{listings}
\usepackage{float}
\usepackage[all]{xy}
\usepackage{tikz}
% new commands
  \newcommand{\QQ}{\mathbb Q}
  \newcommand{\RR}{\mathbb R}
  \newcommand{\CC}{\mathbb C}
  \newcommand{\NN}{\mathbb N}
  \newcommand{\ZZ}{\mathbb Z}
  \newcommand{\DD}{\mathbb D}
  \newcommand{\A}{\mathcal A}
  \newcommand{\B}{\mathcal B}
  \newcommand{\C}{\mathcal C}
  \newcommand{\D}{\mathcal D}
  \newcommand{\R}{\mathcal R}
  \newcommand{\M}{\mathcal M}
  \newcommand{\laplace}{\operatorname\Delta}
  \bibliographystyle{plain}

  \newcommand{\p}{\ensuremath{\mathcal P}\xspace}
  \newcommand{\np}{\ensuremath{\mathcal{NP}}\xspace}
  \newcommand{\fp}{\ensuremath{\mathcal{FP}}\xspace}
  \newcommand{\abs}[1]{\left|#1\right|}
  \newcommand{\sharpp}{\ensuremath{\# \mathcal{P}}\xspace}
  \newcommand{\code}{\texttt}
  \newcommand{\cc}{\code{C++}\xspace}
  \newcommand{\ccx}{\code{C++11}\xspace}
  \newcommand{\irram}{\code{iRRAM}\xspace}
  \newcommand{\MPFR}{\code{MPFR}\xspace}
  \newcommand{\baana}{\code{BA\_ANA}\xspace}
  \newcommand{\anarect}{\code{ANA\_RECT}\xspace}
  \newcommand{\powerseries}{\code{POWERSERIES}\xspace}
  \newcommand{\poly}{\code{POLY}\xspace}
  \newcommand{\func}{\code{FUNC}\xspace}
  \newcommand{\real}{\code{REAL}\xspace}
  \newcommand{\complex}{\code{COMPLEX}\xspace}
  \newcommand{\temp}{\textcolor{red}}
  \newcommand{\seq}{\mathbf}
  \DeclareMathOperator{\lb}{lb}
  \DeclareMathOperator{\bigo}{O}
  \newcommand{\sprec}{prec\xspace}
  \newcommand{\demph}{\textbf}
  \newcommand{\sdzero}{\texttt{0}}
% ntheorem environments
  \theoremseparator{:}
  \theorembodyfont{\itshape}
  \newtheorem{definition}{Definition}[section] 
  \theorembodyfont{\upshape}
  \newtheorem{theorem}{Theorem}[section] 
  \newtheorem{corollary}{Corollary}[section] 
  \newtheorem{example}{Example}[section]
  \newenvironment{proof}{\paragraph{Proof:}}{\hfill$\square$}
\lstset{language={C++},
                basicstyle=\ttfamily,
                keywordstyle=\color{red}\ttfamily,
                stringstyle=\color{blue}\ttfamily,
                commentstyle=\color{green}\ttfamily,
                morecomment=[l][\color{magenta}]{\#}
}
\begin{document}
  \thesistitle{Case Studies in Exact Real Arithmetic - Implementations and empirical Evaluation}{Fallstudien in exakter reeller Arithmetik: Implementation und empirische Evaluation}
  \author{Holger Thies}
  \referee{Prof. Dr. Martin Ziegler}{tba}[tba]
  \department{Mathematics}
  \group{Mathematical Logic}
  \dateofexam{\today}{\today}
  \makethesistitle
  \affidavit{H. Thies}
\dedication{Danksagung...}
\begin{abstract}
    Abstract...
\end{abstract}  
\tableofcontents
  \chapter{Introduction}

  \chapter{Theoretical Background}
  \section{Computers and Real Numbers}
  \include*{tex/computability/computability}
  \include*{tex/complexity/complexity}
  \chapter{Computing with Reals}
  \include*{tex/float/float}
  \include*{tex/multiprec/multiprec}
  \include*{tex/interval/interval}
  \include*{tex/symbolic/symbolic}
  \include*{tex/exact/exact}
  \include*{tex/irram/irram}
  \chapter{Case Study: Dynamic Systems and the Shadowing Lemma}
  \include*{tex/dynamic_systems/dynamic_systems}
  \chapter{Case Study: A Datatype for Analytic Functions}
  \include*{tex/analytic_functions/analytic}
  \chapter{Case Study: Root Refinement for Real Polynomials}
  \section{Introduction}
  \section{Implementation}
  \section{Evaluation}
  \chapter{Conclusion}
  \section{Future Work}
  \bibliography{thesis}
\end{document}
