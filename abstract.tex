
\documentclass[]{article}
\usepackage{hyperref}
  \usepackage[utf8]{inputenc}
  \usepackage[thmmarks]{ntheorem}
  \usepackage{xspace,cleveref,xcolor}
  \usepackage{amsmath}
  \usepackage{amssymb}
  \usepackage{graphicx,url}
  \usepackage[font=small]{caption}
 \usepackage{subcaption}
 \usepackage[parfill]{parskip}
 \usepackage{newclude}
\usepackage{listings}
\usepackage{float}
\usepackage[all]{xy}
\usepackage{tikz}
\usepackage{hhline}

% new commands
  \newcommand{\QQ}{\mathbb Q}
  \newcommand{\RR}{\mathbb R}
  \newcommand{\CC}{\mathbb C}
  \newcommand{\NN}{\mathbb N}
  \newcommand{\ZZ}{\mathbb Z}
  \newcommand{\DD}{\mathbb D}
  \newcommand{\A}{\mathcal A}
  \newcommand{\B}{\mathcal B}
  \newcommand{\C}{\mathcal C}
  \newcommand{\D}{\mathcal D}
  \newcommand{\R}{\mathcal R}
  \newcommand{\M}{\mathcal M}
  \newcommand{\laplace}{\operatorname\Delta}
  \bibliographystyle{alpha}

  \newcommand{\p}{\ensuremath{\mathcal P}\xspace}
  \newcommand{\np}{\ensuremath{\mathcal{NP}}\xspace}
  \newcommand{\fp}{\ensuremath{\mathcal{FP}}\xspace}
  \newcommand{\abs}[1]{\left|#1\right|}
  \newcommand{\sharpp}{\ensuremath{\# \mathcal{P}}\xspace}
  \newcommand{\code}{\texttt}
  \newcommand{\cc}{\code{C++}\xspace}
  \newcommand{\ccx}{\code{C++11}\xspace}
  \newcommand{\irram}{\code{iRRAM}\xspace}
  \newcommand{\MPFR}{\code{MPFR}\xspace}
  \newcommand{\baana}{\code{BA\_ANA}\xspace}
  \newcommand{\anarect}{\code{ANA\_RECT}\xspace}
  \newcommand{\powerseries}{\code{POWERSERIES}\xspace}
  \newcommand{\poly}{\code{POLY}\xspace}
  \newcommand{\func}{\code{FUNC}\xspace}
  \newcommand{\real}{\code{REAL}\xspace}
  \newcommand{\complex}{\code{COMPLEX}\xspace}
  \newcommand{\temp}{\textcolor{red}}
  \newcommand{\seq}{\mathbf}
  \DeclareMathOperator{\lb}{lb}
  \DeclareMathOperator{\bigo}{O}
  \newcommand{\sprec}{prec\xspace}
  \newcommand{\demph}{\textbf}
  \newcommand{\sdzero}{\texttt{0}}
% ntheorem environments
  \renewenvironment{abstract}{%
    \hfill\begin{minipage}{0.95\textwidth}
      \small \textbf{\abstractname.}}
      {\end{minipage}}
  \theoremseparator{:}
  \theorembodyfont{\itshape}
  \newtheorem{definition}{Definition}[section]
  \theorembodyfont{\upshape}
  \newtheorem{theorem}[definition]{Theorem}
  \newtheorem{corollary}[definition]{Corollary}
  \newtheorem{example}[definition]{Example}
  \newenvironment{proof}{\paragraph{Proof:}}{\hfill$\square$}
  \newenvironment{proofsketch}{\paragraph{Proof (Sketch):}}{\hfill$\square$}
\lstset{language={C++},
                basicstyle=\ttfamily,
                keywordstyle=\color{red}\ttfamily,
                stringstyle=\color{blue}\ttfamily,
                commentstyle=\color{green}\ttfamily,
                morecomment=[l][\color{magenta}]{\#}
}
\title{Case Studies in Exact Real Arithmetic - Implementations and empirical
Evaluations}
\author{Holger Thies}
\date{}
\begin{document}
\maketitle
The most common way to represent real numbers on a computer is the use of floating point
arithmetic.
The floating point representation of a real number is a finite word, consisting
of a mantissa and an exponent of fixed length.
Due to the finiteness of the representation, it is unavoidable that errors
occur.
While in many cases the error stays small, there are also several examples
where computations with floating point numbers yield results that are
completely wrong and far off from the correct value.

In any case, the error is not visible from the representation, thus without
additional analysis it is impossible to know if the error is too big for a
particular application or not.
Floating point arithmetic is therefore not a reliable way to perform real number
computations.

This thesis deals with computations on real numbers that are reliable in the
sense that there are algorithms that can compute the result with guaranteed
user-defined error bounds.
This is known as \textbf{exact real arithmetic}.

Computable Analysis gives a sound theoretical foundations for such
computations.
In Computable Analysis, a real number $x$ is called \textbf{computable} if there is
an algorithm (e.g. formalized via Turing-machines) that outputs on input $n$ a rational number $q$ such that
$\abs{x-q} \leq 2^{-n}$. 
By this definition, it is obvious that uncomputable real numbers exist, since
the set of real numbers is uncountable, while the set of algorithms is
countable.

The so called Type-2 Theory of Effectivity (TTE) extends this definition
further, to make it possible to talk about computability and complexity of more
general continuous objects like real functions.

There already exists a vast theory of computability and complexity built on
TTE.
It is an important task to compare the claims made in theory to actual
implementations in exact real arithmetic.

Algorithms for exact real arithmetic can be implemented in modern programming
languages and there are already some frameworks that provide functionality to
simplify such computations.

This thesis uses Norbert M\"{u}ller's \cc package \irram for implementations.
\irram extends \cc by classes and functions for error-free computations with
real numbers.
The most important class is the class \real. An object of type \real is meant
to behave like a  real number that can be manipulated without rounding errors.
Objects of type \real can for example be multiplied, added or subtracted.
An approximation to a real number represented by such an object can also be
printed with any desired precision.
It is guaranteed that the error of the approximation will be small enough to
ensure said precision.
The \irram package takes care of the steps necessary for this. 
It does so by iteratively increasing the precision of the computations.
That is, in a computation instead of a real number $x$ an interval $I = [d-e,
d+e]$ with $x \in I$ is used.
Interval arithmetic is used to bound the error on the result of the performed
operations.
If an approximation to a real number is needed at some point and the precision
does not suffice, the whole computation is started from the beginning with
higher precision, i.e., the intervals are made smaller. 
This process is repeated until the precision is sufficient.

The thesis first gives an introduction to the theory of computability and
complexity on real numbers and an overview of ways to deal with real numbers on
a computer with a detailed view on the \irram package.

It then proceeds by presenting two case studies in exact real arithmetic, i.e.,
two numerical problems are considered in detail from the viewpoint of real
complexity theory and implemented in \cc using the \irram package. 
The results from theory are compared with time measurements obtained from
empirical evaluation.

The first case-study shows how exact real arithmetic can be used to get simpler
algorithms for classical problems in numerical analysis.
The particular problem considered is to compute the shadowing distance for a chaotic
dynamic systems.
The simplest case of a discrete dynamical system is when the state space is 1-dimensional (say $X \subseteq \RR$) and 
the transition function only depends on the previous value,
formally $\Phi(1,x) = f(x)$ for some $f : \RR \to \RR$. 
In this case it can be written by the recurrence relation $x_{n+1} = f(x_n)$ with initial condition $x_0 \in \RR$.
The set of the $x_n$ is called \textbf{orbit} of the map $f$.

When computing an orbit of a chaotic system numerically, i.e., using floating-point
approximations, the computed orbit is usually far off from the real (exactly
computed) orbit. 
A shadowing orbit is a real orbit that stays close to a numerically computed
orbit.
The case study empirically checks how long such shadowing orbits exist
depending on the starting point and on the approximation precision and how
close the numerical and the shadowing orbit are.
The chaotic system considered is the logistic map.

This problem was originally considered by Hammel, Yorke and Grebogi in 1987
\cite{Hammel1987}. 
Since they could not compute real numbers exactly, they used a form of interval
arithmetic to bound the exact orbit.
However, \irram can be used to compute the exact orbit which thus yields a
simplified version of their algorithm.
Additionally a multi-precision arithmetic version is presented using
\code{MPFR}.

The second case-study deals with analytic functions in is based on theoretical
analysis by Kawamura, M\"{u}ller, R\"{o}snick and Ziegler.
Analytic functions have been thoroughly studied in real complexity theory as a
subset of real functions where many in general computationally hard problems become feasible. 
An analytic function is locally defined by its Taylor series.

To show how well those complexity claims compare with practical
implementations, data-types to represent analytic functions have been written.
The implementation is meant as an extension to the \irram framework, that
provides user-friendly classes for computations with analytic functions.
Empirical evaluation was done on the running time of those classes and compared
with the expected running times from the theoretical examination.

\end{document}
